\chapter{Перекрытие.}

Тема перекрытия, как и некоторые другие темы ново-американской школы, была, конечно, известна и композиторам прошлого века, составившим много интересных задач на перекрытие. Но совершенно новое выражение и неожиданное разнообразие приобрела эта тема в руках представителей новой школы.

Тема перекрытия распадается в сущности на две неравноценные группы: перекрытие черными фигурами черных и перекрытие белых фигур (черными или белыми). Наибольший интерес представляет для нас первая группа, на которой мы и остановимся подробнее.

Тема перекрытия черных фигур черными состоит в следующем: защищаясь от грозящего мата или делая просто ход по необходимости, черные сами же перекрывают линию действия черной фигуры (ферзя, ладьи, слона или пешки), -- этим обстоятельством пользуются белые и дают такой мат, который был бы невозможен, если бы черные не перекрыли свою фигуру. 

Поясним на примере.

\begin{center}
 \begin{tabular}{ c }
\textbf{\stepcounter{diagram_counter} № \arabic{diagram_counter}. Г. Венинк.} \\
III пp. <<G.С.>>, март 1919. \\
\chessboard[
\diagramsize,
setfen=1q6/1N2b3/2N1br2/3n4/4P2Q/R1Pk4/B3Rn2/B3K3,
label=false,
showmover=false] \\
\textbf{Мат в 2 хода (1. \knight{}b7--а5).}
\end{tabular}
\end{center} 
		 
Перед нами прекрасная задача Г. Венинка, где показано, как черный конь несколько раз перекрывает черного ферзя. Белые грозят матом 2. Cc4. Черные могут защититься от этой угрозы, отступив куда-нибудь конем d5 и защищая тем самым поле с4 при помощи слона е6. Но куда отступить конем?

На 1. ... \knight{}:сЗ последует просто 2. \rook{}:сЗ\mate{}. На 1. ... \knight{}е3 белые сыграют 2. \rook{}d2\mate{}, пользуясь тем, что поле еЗ заблокировано (см. глава <<Блокирование>>). Остаются еще четыре отступления коня. Рассмотрим каждое из них в отдельности.

Предположим, что черные сыграли 1. ... \knight{}b6. Что произошло? Они перекрыли черного ферзя, отрезав его от поля b1, и теперь белые матуют ходом 2. \bishop{}b1\mate{}. Если черные сыграют 1. ... \knight{}b4, то белые ответят 2. с4\mate{}!, пользуясь тем, что конь перекрыл черного ферзя и отрезал его, от пункта bЗ, а также перекрыл и слона е7. На 1. ... \knight{}с7 белые сыграют 2. \knight{}е5\mate{} (конь перекрыл ферзя по диагонали), а на 1. ... \knight{}f4 2. \queen{}g3\mate{}!, пользуясь тем, что конь одновременно перекрыл и ферзя b8 и ладью f6.

Какой можно сделать общий вывод из этого примера?

Дело, очевидно, не в том, чтобы случайно провести в задаче перекрытие одной черной фигуры другой: такие случайные перекрытия бывали часто и у старых композиторов. Ново-американская школа поставила себе шире цель: добиться в задаче нескольких перекрытий, причем каждый раз (т. е. на каждое новое перекрытие) получается новый мат со стороны белых. В вышеприведенной задаче мы находим четыре таких мата (различных), которые стали возможны лишь в результате перекрытия черных фигур. Композиторы новой школы стали изучать разнообразные комбинации, возникающие из этой темы, и достигли блестящих результатов. Они занялись изучением того, какие фигуры могут быть перекрыты и какие фигуры могут сами перекрывать другие фигуры, -- это значительно обогатило круг комбинаций шахматной композиции.

В задаче № 33 мы видим, что черный конь перекрывает черного ферзя. В задаче № 34 мы находим перекрытие черной ладьи.

\begin{center}
 \begin{tabular}{ c c } 
\textbf{№34. И. Гартонг.} & \textbf{\stepcounter{diagram_counter} № \arabic{diagram_counter}. Д. Венрайт.} \\
<<G.С.>>, октябрь 1919 &  I пр. <<G.С.>>,май 1914 \\
\chessboard[
\diagramsize,
setfen=5nBb/3b2Q1/5r2/7R/3k3n/K7/2P5/4R3,
label=false,
showmover=false] & 
\chessboard[
\diagramsize,
setfen=b7/q1p1N2K/1b1B1kPp/5B2/7P/8/p7/3Q4,
label=false,
showmover=false] \\
\textbf{Мат в 2 хода (1. \king{}b4).} & \textbf{Мат в 2 хода (1. \bishop{}с8).}
 \end{tabular}
\end{center}
	 

Первый ход в задаче № 34 нельзя признать удачным, так как он отнимает у черного короля поле сЗ, но комбинация, проведенная в задаче, очень изящна. Грозит 2. \rook{}d5\mate{}. У черных есть несколько защит: 1. ... \bishop{}с6, но на это последует 2. \queen{}а7\mate{} (ладья f6 теперь перекрыта), 1. ... \bishop{}:f5 2. \queen{}g1\mate{} (снова ладья перекрыта, но по другой линии), 1. .. . \knight{}e6 2. \queen{}:d7\mate{} (опять перекрыта ладья) и, наконец 1. ... \knight{}f5 2. \queen{}g4\mate{}, пользуясь тем, что перекрыты и ладья и слон.

В задаче № 35 мы находим перекрытие обоих черных слонов. Белые грозят матом 2. \knight{}g8\mate{}. Лучшая защита черных — продвинуть пешку с7 и тем связать коня с7. На 1. ... cd последует конечно, 2. \queen{}:d6\mate{}. Но на 1. ... с6 последует 2. \queen{}f3\mate{} (слон a8 перекрыт), a на 1. ... c5 2. \queen{}а1\mate{} (перекрыт слон b6). На первом ходу белые обязательно должны стать слоном на с8, так как при другом отходе слона черные защитятся от мата ходом 1. . . . \queen{}b8!

В задаче № 33 мы видели, как черная фигура порекрывала черного ферзя. Может ли, однако, черный ферзь сам перекрывать другие черные фигуры? На первый взгляд это кажется невозможным, на самом же деле это задание было осуществлено композиторами ново-американской шкоды, и притом неоднократно. Приводим одну из лучших задач на эту тему, где она была впервые выражена (см. № 36).
 
\begin{center}
 \begin{tabular}{ c c } 
\textbf{\stepcounter{diagram_counter} № \arabic{diagram_counter}. Г. Гвиделли} & \textbf{\stepcounter{diagram_counter} № \arabic{diagram_counter}. А. Эллерман.} \\
I пр. <<G. С.>> декабрь 1915. & IV пp. <<G. С.>>, январь 1921. \\
\chessboard[
\diagramsize,
setfen=2r5/4rn2/3N3p/Rq4kP/4bR2/4Q1p1/8/2BB2Kn,
label=false,
showmover=false] & 
\chessboard[
\diagramsize,
setfen=6nB/R7/RK2k3/1pQ1pp2/p3rp2/3b2PB/3qp3/6b1,
label=false,
showmover=false] \\
\textbf{Мат в 2 хода (1. \queen{}d4).} & \textbf{Мат в 2 хода (1. g4).}
 \end{tabular}
\end{center}

Белые грозят матом 2. \queen{}f6\mate{}. У черных есть три любопытные защиты: 1. ... \queen{}c5!, 1. ... \queen{}e5 и 1. ... \queen{}d5. Ho в первом случае черный ферзь перекрывает ладью с8, и белые получают возможность дать мат ходом 2. \rook{}:е4\mate{}. Во втором случае черный ферзь перекрывает ладью е7, и белые матуют ходом 2. \knight{}:е4\mate{}. И, наконец, в третьем случае черный ферзь перекрывает черного слона е4, и белые матуют ходом 2. \queen{}g7\mate{}. Все три тематических варианта этой задачи производят прекрасное впечатление.

В задаче № 37 перекрывающей фигурой явится ладья е4. Белые грозят матом 2. gf\mate{}, -- для защиты oт этой угрозы белым достаточно отступить куда-нибудь ладьёй e4. Но оказывается, что, отступая, эта ладья неизбежно перекрывает какую-нибудь черную фигуру. Действительно, на 1. .. . \rook{}b4 последует 2. \king{}а5\mate{} (перекрыт черный ферзь), на 1. ... \rook{}с4 2. \king{}:b5\mate{} (перекрыт слон d3), на 1. .. . \rook{}d4 2. \queen{}:е5\mate{} (черный слон g1 перекрыт, и белый ферзь получает возможностъ двигаться), на 1. ... \rook{}еЗ 2. \queen{}с6\mate{} (снова перекрыт слон и белый ферзь развязан, но матует уже с другого поля).

Самую любопытную группу перекрытий образуют, однако, те комбинации, которые возникают при взаимном перекрытии черных фигур, например ладей и слонов. Мы познакомим наших читателей со взаимным перекрытием на одном и том же поле слона и ладьи (смотрите задачу № 38). Своим первым ходом белые создали угрозу 2. \queen{}с5\mate{}. Ha 1. ... \bishop{}:d7 последует 2. \bishop{}с7\mate{}, а на 1. ... \rook{}:d7 2. \rook{}g6\mate{}. При рассмотрении этих двух матов мы видим, что они стали возможны потому, что в первом случае слон перекрыл линию ладьи, а во втором -- ладья перекрыла линию слона -- оба раза перекрытие происходит на поле d7. Этих перекрытий белые достигли тем, что поставили на поле пересечения ладьи и слона белую фигуру -- данном случае, коня. В задаче есть еще интересные варианты: 1. ... \rook{}с7 2. \queen{}h6\mate{} и 1. ... \rook{}с6 2. \queen{}:h2\mate{} (<<длинные маты>>).
 
\begin{center}
 \begin{tabular}{ c c } 
\textbf{\stepcounter{diagram_counter} № \arabic{diagram_counter}. Т. Р. Доусон.} & \textbf{№39. E. И. Kyббель.} \\
I пp. <<Western Daily Mercury>>, 1913 &  ,<<G. C.>>, декабрь 1916. \\
\chessboard[
\diagramsize,
setfen=1Nb3B1/r5R1/r2k4/p7/6NB/8/2Q4b/4K3,
label=false,
showmover=false] & 
\chessboard[
\diagramsize,
setfen=7b/5BNb/7r/3n2r1/P1k2N2/Pp1p2Q1/1P5n/3K2B1,
label=false,
showmover=false] \\
\textbf{Max в 2 хода (1.\knight{}d7).} & \textbf{Max в 2 хода (1. \knight{}g6).}
 \end{tabular}
\end{center}
	 
Та же комбинация проведена и в задаче № 39. Своим первым ходом белые становятся на поле пересечения черной ладьи h6 и слона h7, создавая сразу 2 угрозы: 2. \queen{}с7\mate{} и 2. \queen{}f4\mate{}. Hа 1. ... \bishop{}:g6 белые ответят 2. \queen{}с7\mate{}, а на 1. ... \rook{}:g6 2. \queen{}f4\mate{}. Сущность комбинации состоит в том, что, беря белую фигуру, черная фигура тем самым перекрывает линию действия другой черной фигуры (слон порекрывает ладью, а ладья - слона).

\begin{center}
 \begin{tabular}{ c c } 
\textbf{\stepcounter{diagram_counter} № \arabic{diagram_counter}. Э. Палькоска.} & \textbf{\stepcounter{diagram_counter} № \arabic{diagram_counter}. К. М. Григорьев.} \\
I пр. <<G. С.>>, март 1914 & III пр. конк. <<Volk und Zeit>>, 1926. \\
\chessboard[
\diagramsize,
setfen=3rb3/2N5/1PRpp3/NK6/3kp3/BRp3Q1/8/br6,
label=false,
showmover=false] & 
\chessboard[
\diagramsize,
setfen=5n1b/5B2/1nN3p1/1R1b1k1N/8/8/4rPQP/5R1K,
label=false,
showmover=false] \\
\textbf{Max в 2 хода (1. Сс1).} & \textbf{Max в 2 хода (1 h4).}
 \end{tabular}
\end{center}

Эта комбинация носит имя известного композитора Новотного и обычно называется <<темой Новотного>>.

Несколько иную комбинацию мы находим в задаче № 40. Построена она на цугцванг. Интересующую нас комбинацию мы найдем в этой задаче дважды: при взаимном перекрытии черных слонов и ладей на одном и том же поле. Ha 1. ... \bishop{}b2 последует 2. \rook{}b4\mate{}, а на 1. ... \rook{}b2 2. \queen{}: сЗ\mate{}. Здесь мы имеем взаимное перекрытие слона и ладьи на одном и том же поле. Ту же комбинацию мы находим и на другом конце доски: 1. ... \bishop{}d7 2. \queen{}:dб\mate{} и 1. ... \rook{}d7 2. \rook{}с4\mate{}.

Такое взаимное перекрытие на одном и том же поле двух разноходящих черных фигур (т. е. таких фигур, которые движутся различно -- одна по диагонали, а другая по вертикали) получило название <<темы Гримшоу>>.

В задаче № 40 тема Гримшоу проходит вдали от черного короля. В задаче же № 41 поле пересечения ладьи и слона находится вплотную около черного короля. Защищаясь ог угрозы 2. \queen{}fЗ\mate{}, черные стремятся попасть на поле е5, чтобы развязать этим самым слона d5. Ho на 1. ... \rook{}е5 последует 2. \knight{}d4\mate{}, а на 1. ... \bishop{}е5 2. \knight{}е7\mate{}.
 
\begin{center}
 \begin{tabular}{ c c } 
\textbf{\stepcounter{diagram_counter} № \arabic{diagram_counter}. Ф. Дженет.} & \textbf{\stepcounter{diagram_counter} № \arabic{diagram_counter}. В. Б. Райс.} \\
<<G. С.>>‘, май 1916. & <<Pitsburgh Leader>>, 1914. \\
\chessboard[
\diagramsize,
setfen=bbQ5/3p4/8/8/2P1k1p1/1R1N1R2/5B2/3K4,
label=false,
showmover=false] & 
\chessboard[
\diagramsize,
setfen=8/B2p3Q/3pq3/1P1k4/2r3P1/2PN4/4p1N1/4K3,
label=false,
showmover=false] \\
\textbf{Мат в 2 хода (1. \queen{}h8).} & \textbf{Мат в 2 хода (\queen{}h1).}
 \end{tabular}
\end{center} 

В задаче № 42 мы также находим тему Гримшоу, но в ней участвуют не слон и ладья, а слон и пошка. Белые грозят ходом 2. \knight{}с5\mate{}. Hа 1. ... d6 последует 2. \rook{}f4\mate{}, а на 1. ... \bishop{}dб 2. \queen{}:н8\mate{}.

Из этих примеров мы можем выности разницу между темами Новотного и Гримшоу. В теме Новотного на точку пересечения двух разнодвижущихся черных фигур попадает белая	фигура, в теме же Гримшоу это поле пересечения не занято, и черные занимают его одной или другой фигурой, как бы добровольно.
	
Говоря о темах Новотного и Гримшоу нужно указать на то, что возникли они в трехходовой и многоходовой задаче, где имеется больше простора для их проведения, и оттуда уже перекочевали в двухходовку. Поэтому мы касаемся этих тем довольно бегло и намерены еще вернуться к ним во второй части нашей книги.

В задаче № 43 представлена в двухходовой форме "тема Плахутты": она отличается от темы Гримшоу только тем, что на одном поле взаимно перекрываются не разнодвижущиеся, а равнодвижущиеся фигуры, т. е. такие фигуры, которые могут двигаться в одном и том же направлении, напр. две ладьи, ферзь и ладья, ферзь и слон, ферзь и пешка. В задаче № 43 взаимно перекрываются ферзь и ладья на поле е4. 
 
\begin{center}
 \begin{tabular}{ c c } 
\textbf{\stepcounter{diagram_counter} № \arabic{diagram_counter}. И. Гартонг.} & \textbf{\stepcounter{diagram_counter} № \arabic{diagram_counter}. А. Батори.} \\
<<G. С.>>, март 1920. &  I пр. <<G. С.>>, октябрь 1919. \\
\chessboard[
\diagramsize,
setfen=5N2/6p1/2R1r2k/5bR1/1K6/6n1/5Q2/2B5,
label=false,
showmover=false] & 
\chessboard[
\diagramsize,
setfen=4RKN1/1p1p1Pp1/q6p/p1r2k1P/P2PN1Rb/2p3pP/Q1P3P1/8,
label=false,
showmover=false] \\
\textbf{Мат в 2 хода (1. \queen{}а7).} & \textbf{Мат в 2 хода (1. \queen{}с4).}
 \end{tabular}
\end{center}

Белые грозят двойным шахом и матом, от чего черные могут защищаться лишь одним способом: занятием поля е4. Ho на 1. ... \queen{}е4 последует 2. \knight{}f4\mate{} (теперь ладья с4 перекрыта и уже не действует на поле f4, а сам ферзь связан), а на 1. ... \rook{}с4 2. \knight{}еЗ\mate{} (теперь перекрыт и отрезан от поля сЗ черный ферзь, а ладья связана). Провести тему Плахутты в двухходовке чрезвычайно трудно, и задачи с этой темой встречаются единицами.

Остановимся, под конец, на игре черных пешек в теме перекрытия. Мы уже видели, что черные пешки могут и перекрывать другие фигуры и сами могут бьггь перекрыты. Еще один пример того, как пешки могуг быть перекрыты другими фигурами, мы найдем в задаче № 44.

Белые грозят матом 2. \queen{}:g7\mate{}. Ha 1. ... \bishop{}g6 последует 2. \rook{}е5\mate{}! (черная пешка порекрыта, но в то же время развязана ладья е6, которую белые должны отрезать от поля еЗ), на 1. ... \rook{}g6 2. \rook{}:f5\mate{} (снова пешка перекрыта — на этот раз ладьей), а на 1. ... g6 2. \queen{}b7\mate{} (здесь. уже перекрыт слон). Варианты 1. ... g6 и 1. ... \bishop{}g6 образуют, как нам уже известно, тему Гримшоу.

В задаче № 45 черными пешками перекрываются другие черные фигуры: ферзь, ладья и слон. В этой прекрасной задаче проведено 6 таких перекрытий пешками: 1. ... b6 2. \knight{}d6\mate{}, 1. ... b5 2.\queen{}f1\mate{}. -- , 1. ... d6 2. \queen{}e6\mate{}, 1. ... . d5 2. \rook{}е5\mate{}, 1. ... g6 2. \knight:h6\mate{} и 1. ... g5 2. \knight{}с7\mate{}.

До сих пор мы имели дело почти исключительно с такими задачами, где перекрываемая черная фигура свободна в своих движениях. Очень интересные комбинации получаются, однако, когда перокрывается такая черная фигура, которая в начальном положении связана, а после хода черных или при втором ходе белых оказывается свободной. Мы не будем, однако, здесь останавливаться на этом сложном перекрытии, а поговорим о нем подробное в главе ``Новейшие двухходовые идеи''.
