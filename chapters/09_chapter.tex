\chapter{Перемена матов}

Если такие темы, как перекрытие, связывание и развязывание, были уже хорошо известны и композиторам старой
 школы и лишь получили своё настоящее развитие благодаря ново-американской школе, то идея перемены матов 
 целиком принадлежат последнему периоду шахматной композиции. Старая школа знала лишь задачи на Zugzwang, 
 где белые делают чисто-выжидательный ход, в результате которого все маты остаются, а также добавляется 
 один-два мата. В настоящее время мы имеем ряд прекрасных задач, где тема перемены мата получила своё 
 полное выражение.

Задачи на перемену мата можно разбить на две основные группы: задачи с частичной переменой и задачи с 
полной переменой игры. В задачах с частичной переменой матов мы имеем следующую картину: в начальном 
положении на какой-то один или несколько ходов чёрных (напр. на шах с их стороны) у белых имеется готовый 
ответ, но первым своим ходом белые делают этот ответ (один или несколько) невозможным и на указанные выше 
ходы чёрных (в данном случае -- шахи) белые дают мат уже иным способом. Таким образом маты в ответ на 
некоторые ходы черных переменились. Это и есть частичная перемена, т, к. переменились маты не на все ходы 
чёрных, а лишь на некоторые, Конечно, тот ход чёрных, в ответ на который меняется мат, должен чем-нибудь 
выделяться, бросаться в глаза: большей частью это бывает шах со стороны чёрных или отход чёрного короля на 
свободное поле.

Полной переменой мата называется уже иная комбинация: задача построена на полный цугцванг, т. е. на любой 
ход черных у белых уже в начальном положении имеется готовый ответ, но спокойного, выжидательного хода у 
них нет; поэтому они делают такой ход, который меняет всю картину, и в ответ на все или некоторые ходы 
черных последуют уже \so{новые} маты, а не те, которые имелись в начальном положении. Таким образом, в 
задаче на полную перемену мата имеются по существу две позиции: одна т. н. \so{иллюзорная}, та, что 
имеется в начальном положении, и вторая -- \so{действительная}, получающаяся после первого хода белых. 
Большинство матов, заключенных в иллюзорной позиции, вообще даже не осуществляются в задаче, а лишь 
предполагаются. Такие задачи на полную перемену матов мы называем \so{новоматами}.

Для удобства изложения мы познакомим сначала наших читателей с задачами на частичную перемену матов, а 
затем перейдем к новоматам.

В приводимой ниже задаче № 125, сначала даже трудно найти частичную перемену мата -- так она искусно 
замаскирована, но тем не менее она имеется и очень интересна. В начальном положении на ход

\begin{center} 
 \begin{tabular}{ c c }
\textbf{\stepcounter{diagram_counter} № \arabic{diagram_counter}. Г. Гвиделли} & \textbf{\stepcounter{diagram_counter} № \arabic{diagram_counter}. Г. В. Беттман} \\
III пр. <<G. С.>>, март 1916 & I пр. <<G. C.>>, 1918\\
\chessboard[
\diagramsize,
setfen=2R4b/6B1/7r/p6r/K1Bq4/pNkP1R2/5Q1p/1n6,
label=false,
showmover=false]
& 
\chessboard[
\diagramsize,
setfen=5r2/pr1PQ3/1b3N2/4P3/2P4n/R1B1k1Pn/1PP5/1N3K1B,
label=false,
showmover=false] \\
\textbf{Мат в 2 хода (1. \knight{}a1).} & \textbf{Мат в 2 хода (1. e66).}
 \end{tabular}
\end{center}

1. ... \knight{}d2, отрезающий белого ферзя сразу от двух клеток, последует 2. \queen{}xd2\mate{}. Поэтому 
у решающего невольно возникает желание сохранить это расположение фигур. В действительности же этот мат 
отпадает, заменяясь новым: на 1. ... \knight{}d2 последует 2. \queen{}xd4\mate, пользуясь тем, что поле d2 
заблокировано.

Но эта комбинация отнюдь не является типичной для частичной перемены матов. Дело в том, что здесь замена 
матов, в сущности говоря, совершенно не связана с основной темой задачи. В самом деле: центр задачи в 
развязывании белого слона чёрным ферзем, двигающимся по линии своей связки и перекрывающим в то же время 
черные ладьи (1. ... \queen{}e5 2. \bishop{}е6\mate. -- 1. ... \queen{}f6 2. \bishop{}d5\mate). Перемена 
же мата в побочном варианте имеет здесь второстепенное значение.

Типичной же задачей на тему частичной перемены матов является № 126. В начальном положении на 1. ... \rook{}xf6+ последует 2. ef\mate. После первого же хода на 1. ... \rook{}f6+ последует уже 2. Bxf6\mate.  
Грозит 2. \bishop{}e5\mate. Добавочные игры таковы: 1. ... \bishop{}d4 2. \bishop{}d2\mate{}; -- 1. ... 
\rook{}xd7 2. ed\mate{}; -- 1. ... \knight{}f4 2. \knight{}g4\mate{}; -- 1. ... \knight{}f2 2. \knight{}
d5\mate{}. Очень красивая задача.

На чем же основана в этой задаче перемена мата? На том, что матующая фигура (пешка) уводится со своего 
поля, уступая место другой белой фигуре (слону), который и даёт мат. Со стороны черных в этой комбинации 
никакой игры нет: они только снимают белую фигуру с шахом.

В задаче № 127 мы видим уже нечто другое: в главном варианте (то есть в том варианте, где происходит 
замена одного мата другим) мы имеем некоторую игру со стороны черных. В начальном положении на 1. ... e5+ 
последует 2. f5\mate, так как черная пешка заблокировала поле e5. После первого хода 1. ... e5+ последует 
2. \bishop{}e6\mate{}!, пользуясь тем, что пешка e6 перекрыла слона h8. Таким образом, мы имеем здесь не 
только замену одного мата другим, но и замену одной темы другой (блокирования перекрытием). Остальные 
варианты таковы: 1. ... \king{}f5 2. \knight{}g3\mate; -- 1. ... \queen{}d5 2. \bishop{}d3\mate; -- 1. ... \rook{}a2 2. \knight{}d6\mate. Угроза 2. \queen{}b1\mate.

\begin{center} 
 \begin{tabular}{ c c }
\textbf{\stepcounter{diagram_counter} № \arabic{diagram_counter}. С. С. Левман} & \textbf{\stepcounter{diagram_counter} № \arabic{diagram_counter}. С. С. Левман} \\
-- & . --\\
\chessboard[
\diagramsize,
setfen=q1b2Q1b/8/r3p1p1/1N4P1/2B1kP1R/2P4K/4P3/5N2,
label=false,
showmover=false]
& 
\chessboard[
\diagramsize,
setfen=8/2NRp3/Bn1qP2p/2pb3Q/3k1P2/2p2p1K/2P5/4R3,
label=false,
showmover=false] \\
\textbf{Мат в 2 хода (1. \queen{}b4).} & \textbf{Мат в 2 хода (1. \queen{}h4).}
 \end{tabular}
\end{center}

В задаче № 128 приведена трудная тема: перемена мата на шах при полусвязывании.

\begin{center} 
 \begin{tabular}{ c c }
\textbf{\stepcounter{diagram_counter} № \arabic{diagram_counter}. А. Боттаччи} & \textbf{\stepcounter{diagram_counter} № \arabic{diagram_counter}. А. Боттаччи} \\
-- & . --\\
\chessboard[
\diagramsize,
setfen=,
label=false,
showmover=false]
& 
\chessboard[
\diagramsize,
setfen=,
label=false,
showmover=false] \\
\textbf{Мат в 2 хода (1. \k).} & \textbf{Мат в 2 хода (1. \r).}
 \end{tabular}
\end{center}

\begin{center} 
 \begin{tabular}{ c c }
\textbf{\stepcounter{diagram_counter} № \arabic{diagram_counter}. И. И. Зельбергер и П. А. Кетшейд} & \textbf{\stepcounter{diagram_counter} № \arabic{diagram_counter}. А. Эллерман} \\
-- & . --\\
\chessboard[
\diagramsize,
setfen=,
label=false,
showmover=false]
& 
\chessboard[
\diagramsize,
setfen=,
label=false,
showmover=false] \\
\textbf{Мат в 2 хода (1. \k).} & \textbf{Мат в 2 хода (1. \r).}
 \end{tabular}
\end{center}

\begin{center} 
 \begin{tabular}{ c c }
\textbf{\stepcounter{diagram_counter} № \arabic{diagram_counter}. А. Мари} & \textbf{\stepcounter{diagram_counter} № \arabic{diagram_counter}. Д. К. Гейдон} \\
-- & . --\\
\chessboard[
\diagramsize,
setfen=,
label=false,
showmover=false]
& 
\chessboard[
\diagramsize,
setfen=,
label=false,
showmover=false] \\
\textbf{Мат в 2 хода (1. \k).} & \textbf{Мат в 2 хода (1. \r).}
 \end{tabular}
\end{center}
№ 133. A. М а р и.
    III пр. конк. „Alfiere di Re", 1924
Мат в 2 хода (1. ФЬ8).
	№ 134. Д. К. Г е й д о н. .
 I np. ,,G. C.“, ноябрь 1920.
Мат в 2 хода (1. е4).

В задаче А. Мари (№ 133) в начальном положении на  1. ... \queen{}с5+ последует 2. \bishop{}с6X (пользуясь тем, что свободное поле с5 заблокировано), а на  1. ... \bishop{}е6+  2. \knight{}:е6X. Но первый ход, создающий угрозу 2. \bishop{}f2X, делает оба эти мата невозможными. Зато возникает два других, не менее интересных: на  1. ... \queen{}с5+ белые играют  2. \knight{}c6X (поле с5 заблокировано), а на  1. ... \bishop{}е6+ последует  2. \knight{}d7X! (слон перекрыл черную пешку, которая не может попасть на е5). На  1. ... \king{}с5 последует  2. \knight{}d3X.

В № 134 в начальном положении уже существуют два шаха со стороны черных, на которые у белых имеются очевидные ответы:  1. ... \rook{}f5+  2. \rook{}:f5X и  1. ... \rook{}:f3 2. efX. Неожиданный первый ход совершенно меняет ситуацию: теперь на  1. ... \rook{}f5+ белые сыграют уже 1. ... ef (необходимо защищать коня d4, находящегося под боем), а на  1. ... \rook{}:f3 последует  2. \knight{}:f3X, так как поле f5 теперь защищено, белой пешкой. Остальные варианты также хороши:  1. ... \king{}:d4 2. е5X, — 1. ... \bishop{}:d5 2. edX, —  1. ... \knight{}:d4 2. \bishop{}:d6X. Грозит  2. \knight{}c6X.

Чрезвычайно любопытную комбинацию находим мы в следующих двух задачах: помимо перемены одного мата на шах мы имеем здесь, еще перемену перекрытия черной фигуры.

\begin{center} 
 \begin{tabular}{ c c }
\textbf{\stepcounter{diagram_counter} № \arabic{diagram_counter}. А. Эллерман} & \textbf{\stepcounter{diagram_counter} № \arabic{diagram_counter}. К. А. К. Ларсен} \\
-- & . --\\
\chessboard[
\diagramsize,
setfen=,
label=false,
showmover=false]
& 
\chessboard[
\diagramsize,
setfen=,
label=false,
showmover=false] \\
\textbf{Мат в 2 хода (1. \k).} & \textbf{Мат в 2 хода (1. \r).}
 \end{tabular}
\end{center}

№ 135. А. Э л л е р м а н.
II поч. отз. „G. С.‘‘, январь 1920
Мат в 2 хода (1. КеЗ).	№ 136. К. A. К. Л а р с. е н.
. I пр. „G. С.“, декабрь 1920,
Мат в 2 хода (1. е4).


    В № 135 на 1. ... е5+ последует 2. Кеб, пользуясь тсм, ч-го пешка перекрыла ладью е2. Но первым ходом белые жертвуют коіш на еЗ (грозит 2. Фе4), что лишает их возможносги провести в ответ иа ша* прежний маневр. Но аато тсперь на 1. ... с5 + последует 2. Kf5 в вид?
Taro, что мшка перекрыла ладыо л5. Таким образом, порсмсна мата ?рганически связана с ааменой сдкого порекрыгия другим. На 1. .. • ?Л: еЗ послодусг 2. Фаі, на 1. ... Кре5 2. Ф1і8, а на 1. ... Лс5
2. Кс2.
    Ещо ТОКЬІПС, пожалуй, происдспа эта идея в задачо № 136. В на- чальном положскии на 1. ... f3-t- имеется ужс готовый отвст 2. Кс15 '(бслые пользуются тсм, что пешка перекрыла слона Ы. После первого лода, на 1. ... fc+ белыо сыграют уже 2. Фс14, пользупсь псрскрытиом ?слоча gl. Остальныо варианты таковы: 1. ... Ко.З 2. Фс4,—-1. ... К: Ь6
2. Л:а5—1. ... dc 2. КЬ7,—1. ... Л: аб 2. ФЬ5. Грозит 2. Фа5.

\begin{center} 
 \begin{tabular}{ c c }
\textbf{\stepcounter{diagram_counter} № \arabic{diagram_counter}. И. Гартонг} & \textbf{\stepcounter{diagram_counter} № \arabic{diagram_counter}. А. Мари} \\
-- & . --\\
\chessboard[
\diagramsize,
setfen=,
label=false,
showmover=false]
& 
\chessboard[
\diagramsize,
setfen=,
label=false,
showmover=false] \\
\textbf{Мат в 2 хода (1. \k).} & \textbf{Мат в 2 хода (1. \r).}
 \end{tabular}
\end{center}

№ 137. И. Г а р т онг.
I пр. конк. шахм. клуба в Враиау, 1925.
Мат в 2 хода (1. ФЬ2).
	№ 138. A. М а р и.
Н по. кокк. „Е1 Ajedrez Argcntino“, 1926.
Mar в 2 хода (1. Феі).


    Б зпклгочеиио мы приподим дпо задачи, гдс проводоно три иеромсиы Матов па яіахи: покамсот это количосгво нужко признагь рскордным.
    В № 137 на 1. ... Фа6+ у белых ссгь уже готовый отвст 2. Сс4, Sa 1. ... ФЬЗ -+? 2. СгЗ, а на 1. ... Ф?54- 2. Се4. После первого хода белых, отдающого чорному королю поле f6, эти маты уже нсвозможны. 'Они заменяюгся другими: 1. ... Фаб + 2. КЬ5.—1. ... ®f54- 2. К : f5.—
1. ... ФЬЗ-І- 2. Kf3. Ha 1. ... Kpf6 послсдусг 2. Кеб. В №138 ме- йяюгся маты в огваг на шахи 1. .. . Фгб, 1. . . . Феб и 1. ... Ф : #5.
    На ‘ этом мы закончим рассмотронио задач с часгичной перемсион олатов. Мы считаем нсобходимым указать, что именно в этой области ясрод хомпозигорами новой школы открыгы большио просгоры и воз- можяости, так как эта тема таит богатейшие источники красивых и ?бффсктных комбинаций.

Переходим теперь к рассмотрению задач на полную перемену матов, т. е. новоматов.

Мы считаем необходимым упомянуть о том, чго предшествсштками ?йовоматов, несомненно, были те задачи на цугцванг, когорыо были столь излюбленньг старой английской школой. Правда, очснь часго эти задачи йа Zugzwang не содержали никакой яркой кдои: нужно было наііти какой-нибудь чисго-выжидательный ход—и только. О замено одкого мата другим ко было речи. Но иногда эгот чисто-выжидательиый ход было очакь трудно найти. Привсдем дво задачи из числа лучших задач на йоляык Zugzwang, гдо вся трудность заключаегся кменно в выборо пер- soro, чисго-выжкдатольного хода.

\begin{center} 
 \begin{tabular}{ c c }
\textbf{\stepcounter{diagram_counter} № \arabic{diagram_counter}. А. Кремер} & \textbf{\stepcounter{diagram_counter} № \arabic{diagram_counter}. П. Г. Вильямс} \\
-- & . --\\
\chessboard[
\diagramsize,
setfen=,
label=false,
showmover=false]
& 
\chessboard[
\diagramsize,
setfen=,
label=false,
showmover=false] \\
\textbf{Мат в 2 хода (1. \k).} & \textbf{Мат в 2 хода (1. \r).}
 \end{tabular}
\end{center}


\begin{center} 
 \begin{tabular}{ c c }
\textbf{\stepcounter{diagram_counter} № \arabic{diagram_counter}. К. Херлей и К. Ветней} & \textbf{\stepcounter{diagram_counter} № \arabic{diagram_counter}. Г. Е. Функ} \\
-- & . --\\
\chessboard[
\diagramsize,
setfen=,
label=false,
showmover=false]
& 
\chessboard[
\diagramsize,
setfen=,
label=false,
showmover=false] \\
\textbf{Мат в 2 хода (1. \k).} & \textbf{Мат в 2 хода (1. \r).}
 \end{tabular}
\end{center}
?Іат a 2 хода (1. Фс8).	№ 142, Г. Е. Ф у н к.
I up. „G. С.“. апролі> 1922.
Wax в 2 хода (1. Ф&1]

Здесь идея новоматов представлопа, конечно, п самой примигииной форме, іго ведь и материалу-то на доске не так уж много!

В № 142 мы имеем яолную перемону матов в более развернутом виде. В начальном положении на 1. ... с5 последует 2. ФЬ7, на 1. ... сіб 2. Ф : сб, на 1. . .. 2. ФсІ4 и иа 1. ... {5 2. Ло5. После ггервого хода белых последиие два мата остаются, но первые дца уже не прохо-
с5 последует 2. Ф&8, а на 1.
Гораздо более яркую идею находим мы в задаче № 143. В началь- ком положении всс очень просто: на 1. ... Ф '? аі белыо могут сыграть ' 2. Леб, на 1. ... d5 2. Ф : о5, а на 1. .. . 2. Af3. Эффектный пер- вый ход создает совершенно новую игру для черных: 1. ... Фс!5 (е4)+- -2. Af3.—1. ... Фа4! 2. С : d4.-l. ... d5 2. Лс6.-1. ... Ф : сЗ 2. Ф : сЗ. Оказызается, что в атом свободном положемии у белых есгь только одиа
№ 143. Г. Д о б б с.
I пр. „G. С.“, фсвраль 1915.
Мат в 2 хода (1. ЛсЗ)	144. С. Гертма и.
I пр. Брит. Шахм. Союза Проблем., 1926.
Мат в 2 хода (1. Kf3).

-воаможность дать мат'в 2 хода—развязать черного ферзя.

\begin{center} 
 \begin{tabular}{ c c }
\textbf{\stepcounter{diagram_counter} № \arabic{diagram_counter}. Г. Доббс} & \textbf{\stepcounter{diagram_counter} № \arabic{diagram_counter}. С. Гертман} \\
-- & . --\\
\chessboard[
\diagramsize,
setfen=,
label=false,
showmover=false]
& 
\chessboard[
\diagramsize,
setfen=,
label=false,
showmover=false] \\
\textbf{Мат в 2 хода (1. \k).} & \textbf{Мат в 2 хода (1. \r).}
 \end{tabular}
\end{center}

Во всех приведеапых нами примерах Zugzwnng имеется по толысо в начальном положении, но к после 1-го хода белых. Белыб ничем не грозят и только ждут хода черных, чтобы получить возможность для мата. Но егть и такие ковоматы, где первоначальный полный Zugzwang заменяется после 1-го хода белых угрозой, как видно из задачи № 144. В начальдой позиции на 1. ... Ф : g-4 последует 2. A : g4, на 1. ... с : g3 2. К : дЗ, на 1. ... К : d2-f- 2. К : d2, а на любое другое отступление коня 2. ЛоЗ. Первый ход уничтсжает все эти маты и в то же премя соэдает угрозу 2. Ф : с4, которая проходиг, прпида, лишь прн ходе
1. ... Ф : g4. Ha 1. ... С : g3 белме играют 2. К : g5, на 1. ... К : (12 4-
2. Kf : с12, на 1. ... К : е5 2. Ф : е5, а на другие отходы коня 2. ФсіЗ. В данной яадачо угроза проходит лшмь в одиом случао, но бывают ноио- маты, где угроаа проходит в нескольких вариантах.
К"іЗ Посмотрим тепорь, какие идеи могут быть проведены в форме новоматов.

\begin{center} 
 \begin{tabular}{ c c }
\textbf{\stepcounter{diagram_counter} № \arabic{diagram_counter}. В. Б. Райс} & \textbf{\stepcounter{diagram_counter} № \arabic{diagram_counter}. И. Гартонг} \\
-- & . --\\
\chessboard[
\diagramsize,
setfen=,
label=false,
showmover=false]
& 
\chessboard[
\diagramsize,
setfen=,
label=false,
showmover=false] \\
\textbf{Мат в 2 хода (1. \k).} & \textbf{Мат в 2 хода (1. \r).}
 \end{tabular}
\end{center}

№ 145. В. Б. Р а й с.
III поч. огз. „G. С.“, фепраль 1915.
Мат в 2 хода (1. Саі).
	№ 146. И Г а р т о н г.
I up. „G. С.“, агіроль 1922.
Мат в 2 хода (1. Фс8).


    Вот пред нами задача (№ 145), где в форме новомата проведена труднейшая идея освобождения линии для белой фигуры, т. н. б р и- стольская идея'. В начальной позиции на 1. ... g6 последует 2. С : с7, а на любое движение ладьи—взятие ее с матом. После же 1-го хода на 1. ... g6 послсдует 2. ФЬ2!
    В № 146 полная персмена матов происходи? на фоне превращенпя черной пешки. В начальном положонші на 1. ... сІФ послодуст 2. Фі;2, а на 1. ... el К 2. Фі2 (ііонтральныо варианты), на 1. ... К : ЬЗ
2. Ф : е2, на 1. ... g5 2. Фі5, а на 1. ... К<У) 2. Фе4. Пог.ле велнко- лепного первого хода вся игра мекяется: теперь на 1. . . . еІФ после- дует 2. Cg2! а на 1. ... еІК 2. Af2. Кроме того, меняются еще 2 мата:
1. ... К : ЬЗ 2. Ф : ЬЗ и 1. ... КіЛ 2. Фх4.
    В новомате можно провести не тодько связыванио и развязывавие фигур, но и перемену того и другого, а также перемеігу перокрытий. Возьмем для примера задачу № 147. В ней с иэумителыіым мастерством проводона труднсйшая идоя перомоны перокрытий в новомаге.

\begin{center} 
 \begin{tabular}{ c c }
\textbf{\stepcounter{diagram_counter} № \arabic{diagram_counter}. А. Уайт} & \textbf{\stepcounter{diagram_counter} № \arabic{diagram_counter}. Г. Гвиделли} \\
-- & . --\\
\chessboard[
\diagramsize,
setfen=,
label=false,
showmover=false]
& 
\chessboard[
\diagramsize,
setfen=,
label=false,
showmover=false] \\
\textbf{Мат в 2 хода (1. \k).} & \textbf{Мат в 2 хода (1. \r).}
 \end{tabular}
\end{center}

№ 148. Г. Г в н д с л л н.
IV rip. „G. С.“, январь 1917.
Мат в 2 хода (1. КрсЗ).	Мат в 2 хода (1. Са4),

1 О „бристольекой" идее см. подробнев в главе „Первый ход белых“.

    В этой задачс, в начальном положснии, на 1. ... Cg3 [послсдусг
2. ФеЗ (идонный вариант), посло 1-го жо хода бслых на 1. ... С;;3 но- следуст 2. Кс5! Слон н в том и в другом случаях перекрыпает ладыо, по аііачснио и реаультат этого порокрытия восьма рпалнчиы. В япдачо ость и сщо псремсиа матов: 1. ... ЛсИ 2, Ф : <14 (пмосго 2. Of5).—
1. ... ЛсЗ 2. Of5 (иместо 2. Ф : оЗ). ГІсриый ход болых, моняющий лкнига связки белого коня, также очень интсресеи.
    В аадачс № 148 проведсна замена матов на открытый шах белому королю,—эта идоя чреавьічайно трудно осуществима в нопомате. В на- чальном положсшш иа 1. . .. Ь4+ бслые могут отво?игь 2. КЬс4. Посло ж.о 1-го хода на 1. ... Ь4+ получается мат другим конем 2. Kdc4. Мат жо
2. КЬс4 проходит в варианте 1. ... Ьа+.
Можко ли, одкако, провости в ковоматс комбинацию полусвязки? Оказывается, можно, и не только комбинацию полусвязки, но и за- мсну одной полусвяаки другой. В задачс № 149 в начальной (иллюзор- нок) позкцик комбинация полусвявки не проходит, но зато она рсльефно выступает в декетвитсльной игро. Дсйствителвно в начальном положении

\begin{center} 
 \begin{tabular}{ c c }
\textbf{\stepcounter{diagram_counter} № \arabic{diagram_counter}. Л. А. Исаев} & \textbf{\stepcounter{diagram_counter} № \arabic{diagram_counter}. К. А. Ларсен} \\
-- & . --\\
\chessboard[
\diagramsize,
setfen=,
label=false,
showmover=false]
& 
\chessboard[
\diagramsize,
setfen=,
label=false,
showmover=false] \\
\textbf{Мат в 2 хода (1. \k).} & \textbf{Мат в 2 хода (1. \r).}
 \end{tabular}
\end{center}

№ 149. Л. А. И с а е в.
III і;р. конк. „Brit. Chess РгоЫ. Soc.“, 1926.
	№ 150. К, A. К. Л а р с е н. 
I пр. „G. С.“, апрсль 1921.


на 1. ... d3 (или dc) послодуот 2. A: е4, а на 1. ... оЗ 2. Kd3. Первый ход бслых моняот ситуацию: тепсрь на 1. ... dc (или d3) послсдует
2. Kd3 (полусвяяка), а іш 1. ... сЗ 2. Ф : оЗ (тожо полусвяака). Ипто- рссная псрсмона мата получпстся такжо нри amjjm'O 1. ... j{5: и иллю- аорной ыгрс—2. Ф : с4, а в действитсльной—2. К^2!
    В задачс же № 150 мы имсем замену одной систомы волусвязки другой. В начальном положекии на 1. ... d6-t- балые могу? играть
2. cd (полусвязка), а на 1. ... сб 2. Фсб (вторая полусвязка). Послс . 1-го хода белых, отдагощсго черному королю свободноо полс d8, на 1. ... d6+ цоеледуст у.чіо 2. Ф : с\6 (1-ая полусвяака), а на 1. ... сб 2. К : еб Заликолепная и единственная позса на эту тему
(/-ая полу задача!
     Мы видим, такнм образом, что к в форме новоматов могуг быгь про- водены труднейшио и интереснейшие идси ново-американской школы. Но у новоматов ссть и свои собствснкыо идеи, с которыма мы и хотим п оз е&к омк':'ь ч итатслей.
     Во-псрвых, идея соотзетствуюідих п о л с й. Чтб гіредсгавляст ссбою этк кдси, ііы поймем на примере ярекрасной задачи Е. И. Куб- \j\2 151).

\begin{center} 
 \begin{tabular}{ c c }
\textbf{\stepcounter{diagram_counter} № \arabic{diagram_counter}. Е. И. Куббель} & \textbf{\stepcounter{diagram_counter} № \arabic{diagram_counter}. Б. Д. Эндред} \\
-- & . --\\
\chessboard[
\diagramsize,
setfen=,
label=false,
showmover=false]
& 
\chessboard[
\diagramsize,
setfen=,
label=false,
showmover=false] \\
\textbf{Мат в 2 хода (1. \k).} & \textbf{Мат в 2 хода (1. \r).}
 \end{tabular}
\end{center}

\begin{center} 
 \begin{tabular}{ c c }
\textbf{\stepcounter{diagram_counter} № \arabic{diagram_counter}. А. Д. Финк} & \textbf{\stepcounter{diagram_counter} № \arabic{diagram_counter}. Ф. Дженет} \\
-- & . --\\
\chessboard[
\diagramsize,
setfen=,
label=false,
showmover=false]
& 
\chessboard[
\diagramsize,
setfen=,
label=false,
showmover=false] \\
\textbf{Мат в 2 хода (1. \k).} & \textbf{Мат в 2 хода (1. \r).}
 \end{tabular}
\end{center}

В задачс № 153 мы находим очень удачнос осуш,сствленио этой идеи. В начальном хголожении на 1. ... Ф : ЬЗ последует 2. Ла5, на 1. . .. ФЬ4
2. ЛЬ5 : Ь4, а на 1. ... Ф : Ь5 2. Ф : Ь5. Но иервый ход болых порсно- сит и?ру с ворхпой полотты доеки іт тіжшок), ц ііолучлатои «'лодую" iu.ce: иа 1. ... Ф : Ь5 белмо нгршох 2. ЛаЗ, на 1. ... ФЬ4 2. ЛЬЗ : Ь4 и на 1. ... Ф : ЬЗ 2. Ф : ЬЗ. Остальныс жо маты остаются без изменекия.
    Теперь мы переходим к той группе новоматов, в которых центр тяжести заключается не столько в замене старых матов новыми, сколько в создании cnje новых матов—дополпитслыю к прегкним. Есть и такие новоматы, гдо заменм матов вовсо нет, а вее сводится к соадавию до- полнитсльных матов. ГІримср такой задачи мы приводим под № 154.

\begin{center} 
 \begin{tabular}{ c c }
\textbf{\stepcounter{diagram_counter} № \arabic{diagram_counter}. К. Промысло} & \textbf{\stepcounter{diagram_counter} № \arabic{diagram_counter}. Б. Залкинд} \\
-- & . --\\
\chessboard[
\diagramsize,
setfen=,
label=false,
showmover=false]
& 
\chessboard[
\diagramsize,
setfen=,
label=false,
showmover=false] \\
\textbf{Мат в 2 хода (1. \k).} & \textbf{Мат в 2 хода (1. \r).}
 \end{tabular}
\end{center}

№ 155. К. Пр омысло.
II пр. „G. С“, декабрь 1914.
Мат в 2 хода (1. Леб).
	

В этой вадачо на любой ход слонов или коней король отходит с ма- том за вскрышку. Ha 1. ... d5 белые сыграют 2. Фе2. Своим иервым ходом белые не заменяют ни одного мата, но создают два новых, так как червый король получает два свободных поля. Ha 1. ... КреЗ после- дуот 2. ®d3, а на 1. ... Кре5 2. Фгі4. Остальные маты остаются без измснония.
    Однако большинство новоматов на эту тему содержит также псрс- моку одкого или нсскольких матов. Пркведем несколько нримеров таких аадач.
    Б задачо Лга 155 у чорііых нсмного ходов. Ha 1. ... Ксл іюслодуот
2. ЛК4, на 1. ... Cjj5 2. Ф : g5, а на 1. ... Ь2 2. Kg2. Своим первым ходом белыо создают такую позицию, в когорой, во-первых, меняюгся 3 мата и создаются 2 дополнигсльных. Ha 1. ... К : d3 последует те- перь 2. К : аЗ, на 1. .. . Ко") 2. Ло4, на 1. . . Cg5 2. ©f3 (нерсменившиеся маты), на 1. ... g6 2. Af6 и 1. ... g-5 2. ФП (дополнительяые каты).
    Exije более эффсктна задача № 156. Здесь в начальном положении могкст двкгаться лишь одна чсрная фигура—слон, при любом отстугілении которого бслые дают мат 2. Ксб. В результате 1-го хода этот мат без- воззратпо отпадасх, но зато возникаст целых 7 поі.ых матові А именпо: 1..... СсЗ 2. be.—1.'... СсЗ 2. fc.—1. ... Са5 (Ь4) 2. КЬ4—1. ... СІіб
2. Kf4.—1. ... Ссі 2. К : cl.—1. ... Cel 2. К : еі и, наконец, 1. ... Хр : йЗ 2. Л : d2. Любопытиейшая позиция!
    A вот eige один любопытный прикер такого новомач’а (№ 157): в на- чальном полоаении аа 1. .. . с4 последует 2. С : d5, на 1. ... Лс/1
2. Л : Ь6, на 1. ... С : g7 2. fg.—1. ... С : f6 2. Ф : f6 и т. д. Псрвым сбоим ходом бсльіс мсняют ответ на ход 1. ... с4 (теперь іюследует
2. Л : с4 и создают 2 новых мата: 1. ... сЬ 2. Фсі и 1. .. . Ь5 2. ab.

\begin{center} 
 \begin{tabular}{ c c }
\textbf{\stepcounter{diagram_counter} № \arabic{diagram_counter}. А. Шимай-Мольгар} & \textbf{\stepcounter{diagram_counter} № \arabic{diagram_counter}. Л. Шор и Ю. Нейкомм} \\
-- & . --\\
\chessboard[
\diagramsize,
setfen=,
label=false,
showmover=false]
& 
\chessboard[
\diagramsize,
setfen=,
label=false,
showmover=false] \\
\textbf{Мат в 2 хода (1. \k).} & \textbf{Мат в 2 хода (1. \r).}
 \end{tabular}
\end{center}

№ 157. А. Ш и м a й-М о л ь г a р.
I пр. конк. „Chcmnitzer Taj/oblatt". 
Мат в 2 хода (1. ЛЬ4).
	№ 158. Л. Ш о р и Ю. Н е й к о м м.
Мат в 2 хода (1 Kf5).

В заключение приведем еще несколько задач на полную перемену матов, в которых показаны современные достижения задачной техники. Мы считаем, однако, необходимым отметить, что в последние годы замечается упадок новоматов. Все меньше мы встречаем таких новоматов, где помимо перемены матов проводится еще и какая-нибудь яркая стратегическая идея. В большинстве случаев мы имеем в этих задачах механическую замену нескольких обычных матов другими матами — столь же обычными и неинтересными. Особенно много таких бесцветных задач составляют венгерские проблемисты, — так что можно даже говорить о ``венгерской школе'' новоматов. Время от времени на международных конкурсах попадаются, однако, и более яркие новоматы, свидетельствующие о том, что и в этой области шахматной композиции возможны новые достижения.

Задача № 158 принадлежит двум виднейшим венгерским проблемистам, из которых Ю. Р. Нойком известен к тому же, как крупнейшей знаток двухходовой задачи вообще и новоматов — в частности. В начальном положении белые имеют готовые маты на любой ход черных: 1. ... Лс/5 2. Cf5.—1. ... Сс15 2. Ф1і7.—1. . .. Ксл 2. Л : еЗ. После прекрасного первого хода, отдающего королю свободное поле f5, все эти маты меняются: 1. ... Лсл 2. Л : <14.—1. ... Л : (16 2. К : (16.—1. ... С(І5 2. Ф^4.—-1. ... Ксл 2. Kg3. Кроме того, добавился новый мат: 1. ... Кр : f5 2. Фй6.

\begin{center} 
 \begin{tabular}{ c c }
\textbf{\stepcounter{diagram_counter} № \arabic{diagram_counter}. И. Олас} & \textbf{\stepcounter{diagram_counter} № \arabic{diagram_counter}. С. Гертман} \\
-- & . --\\
\chessboard[
\diagramsize,
setfen=,
label=false,
showmover=false]
& 
\chessboard[
\diagramsize,
setfen=,
label=false,
showmover=false] \\
\textbf{Мат в 2 хода (1. \k).} & \textbf{Мат в 2 хода (1. \r).}
 \end{tabular}
\end{center}

№ 159. И. О л а с.
Ill	пр. „Magyar Sakkviiag", 1925.

Mar a 2 хода (1. Фе4).
	

Очень легка по построению и забавна по своям вариантам задача № 159. В начальном положении у черных всего лишь один ход 1. ... hg на что может последовать 2. knight{}f1\mate{}. После первого хода не только меняется этот ответ (1. ... hg 2. \queen{}xg4\mate{}) но и создается ряд новых матов, а именно: 1. ... \knight{}g5 2. \knight{}f3\mate{}. -- 1. ... \king{}g3 (h3) 2. \knight{}f1\mate{}.

В задаче № 160, принадлежащей одному из молодых и очень талантливых венгерских проблемистов, приведена замена четырех матов пятью другими с добавлением нового варианта. В начальном положении на. 1. ... \knight{}c1~ 2. \queen{}d3\mate{}. -- 1. ... \knight{}e2~ 2. \queen{}c3\mate{}, -- 1. ... \rook{}~ 2. \bishop{}d5\mate{}. 1. ... \bishop{}с5. После первого хода белых на 1. ... \knight{}c1-а2 последует 2. queen{}xb3\mate{}, на 1. ... \king{}d3 2. \queen{}e6\mate{}. -- 1. ... \knight{}е2~ 2. \queen{}d4\mate{}. -- 1. ... \rook{}~ 2. \rook{}b4\mate{}. -- 1. ... \bishop{}xс5 2. \rook{}xс5\mate{} и на 1. ... \king{}d5 2. \queen{}e6\mate{}.
