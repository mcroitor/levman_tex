\chapter{Второй ход белых (мат).}

В двухходовой задаче второй ход белых является матующим. Позиция, получающаяся после второго хода белых, называется либо просто \so{матом}, либо \so{матовой картиной}.

Ко второму ходу белых, так же как и к первому, предъявляется ряд требований, - некоторые из них мы считаем формальными, другие же несомненно являются идейными. Скажем прежде всего несколько слов о формальных требованиях, предъявляемых ко второму ходу белых.

Матующий ход белых должен быть, по возможности, более скрытым, замаскированным, - это затрудняет решение. Нужно избегать на матующем ходу забирать черную фигуру (пешки в счет не идут), - так как это выглядит очень грубо. Исключения допускаются, но лишь в отношении побочных малозначащих вариантах или тех случаев, когда битье черной фигуры непосредственно входит в тему задачи. Безусловно обязательно, чтобы в главных (идейных) вариантах на каждый ход черных у белых был только один определенный матующий ход. Если на какой-нибудь ход черных белые могут матовать несколькими способами, по-разному, то в задаче получается дуаль (одна или несколько). Насчет дуалей существуют различные мнения. Английские авторы, например, считают недопустимой какую бы то ни было дуаль в задаче – даже в побочном варианте. Большинство проблемистов не придают, однако, дуали такого значения, если только она не попадается в главных вариантах.

Мат может быть прямой, когда белая фигура непосредственно нападает на короля и тому некуда укрыться. Мат может получиться также в результате того, что какая-то белая фигура отходит, открывая стоящую позади нее дальнобойную фигуру, -- такой мат будет называться батарейным. Если мат объявляет слон (или ферзь по диагонали), то такой мат называют \so{диагональным}. Мат же ладьей (или ферзем по вертикали или горизонтали) называется \so{фронтальным}.

Художественная (или иначе -- чешская) школа выдвинула и обосновала понятие \so{чистого} и \so{правильного} мата. Чистым называется такой мат, когда каждая клетка вокруг черного короля либо занята черными фигурами, либо охраняется какой-нибудь одной белой фигурой. Если на соседнее с королем поле бьют одновременно две белых фигуры, то такой мат будет уже нечистым. Нечистым будет также и такой мат, когда какое-нибудь соседнее с черным королем поле одновременно и занято какой-нибудь черной фигурой и находится под ударом белой фигуры. При этом, если в чистой матовой картине принимают активное участие все имеющиеся в задаче белые фигуры (исключение допускается лишь в отношении белого короля и пешек), то такой мат принято называть еще и правильным. Если в матовой картине имеется какая-нибудь связанная черная фигура, то мат будет чистым лишь в том случае, если связанная черная фигура оказывает известное влияние на матующую фигуру, то есть, иными словами, если мат проходит лишь благодаря связанности этой фигуры, -- будь эта фигура свободна, мат не проходил бы. Поясним примерами.

\begin{center}
 \begin{tabular}{ c c }
\textbf{№ 23. М. Фейгль.} & \textbf{№ 24. С. Лойд.} \\
I пр. <<Tidskrift>>, 1904 & <<Wilkes Spirit of the Times>>, 1867. \\
\chessboard[
\diagramsize,
setfen=n1b3NK/8/8/2QPk1p1/r2nB1RP/8/3N1P2/8,
label=false,
showmover=false]
& 
\chessboard[
\diagramsize,
setfen=8/5N2/4p3/2K5/4k3/7R/3B2Rn/3b3Q,
label=false,
showmover=false] \\
\textbf{Мат в 2 хода (1. \queen{}аЗ)} & \textbf{Мат в 2 хода (1. \rook{}f2).}
\end{tabular}
\end{center}

В задаче № 23 мы находим несколько правильных матов. Например, на 1. ... \rook{}:аЗ белые отвечают 2. \knight{}с4\mate. При изучении получившейся матовой картины мы видим, что мат чистый: каждое из полей вокруг черного короля охраняется лишь одной белой фигурой (поля d5 и f5 -- слоном, e4 и f4 -- ладьей, d6 и f6 -- конями и сб -- пешкой), поле же d4, занятое черным конем, не атаковано белыми. Но этот мат еще и правильный потому, что в мате принимают непосредственное участие все белые фигуры (белый король, а также пешки f2 и h4 в счет не идут). Правильные маты получаются также в вариантах 1. ... gh 2. f4\mate и 1. ... \knight{}~ 2. \knight{}f3\mate. При проведении угрозы 2. \queen{}g3\mate мат, вообще говоря, получается нечистый, но если черные снимут ладью слоном, то и этот мат будет чистым и даже правильным. Таким образом, в этой задаче мы насчитали целых четыре правильных мата.

В задаче № 24 мы находим несколько чистых матов, но всего лишь один правильный. Правильный мат получается лишь в варианте 1. ... \king{}:f3 2. \king{}g5\mate. В вариантах 1. ... \bishop{}:f3 2. \queen{}b1\mate и 1. ... \knight{}:fЗ 2. \queen{}h7\mate маты получаются чистые, но так как белая ладья совершенно в них но участвует, то они не являются правильными.

\begin{center}
 \begin{tabular}{ c c }
\textbf{№ 25. И. Шелль.} & \textbf{№ 26. З. Мах.} \\
<<Casopis>>, 1919. & <<Nove Mody>>, 1867. \\
\chessboard[
\diagramsize,
setfen=8/B6r/k5p1/1R6/8/8/KQ6/8,
label=false,
showmover=false]
& 
\chessboard[
\diagramsize,
setfen=K7/7Q/2R2P2/2nkP3/5R2/P7/4P3/8,
label=false,
showmover=false] \\
\textbf{Мат в 2 хода (1. \rook{}b5)} & \textbf{Мат в 2 хода (1. \rook{}f2).}
\end{tabular}
\end{center}

Рассмотрим еще маты, получающиеся в задаче № 25. На 1. ... \rook{}:h5 последует 2. \queen{}b6\mate (угроза, содержащая правильный мат), а на 1. ... \king{}:а7 2. \rook{}а5\mate (снова правильный мат). Но в задаче есть и еще один правильный мат 1. ... \rook{}:а7 2. \queen{}b5\mate. На первый взгляд может показаться, что поле а5 находится одновременно под ударами двух белых фигур -- ферзя и ладьи, но эхо лишь обман зрения: ладья держит лишь поле b5, а через белую фигуру она влиять не может.

В задаче № 26 мы встречаем правильные маты при связанной черной фигуре. На 1. ... \king{}:с4 последует 2. \queen{}dЗ\mate, а на 1. ... \king{}:с6 2. \queen{}b7\mate. Эти оба мата правильные, так как в обоих случаях момент связки черного коня имеет существенное значение при мате. То обстоятельство, что поле с5 одновременно занято черным конем и атаковано белой фигурой, не нарушает чистоты матовой картины. В задаче есть и еще один правильный мат 1. ... \king{}:е5 2. \queen{}h5\mate.

\textbf{Эхо-мат.} Эхо-матами называется такая группа матов в одной задаче, которые полностью сходны между собой по матовой картине, т. е. поля вокруг черного короля в одном случае заняты или защищены так же, как и в другом. При этом предполагается, что король получает во всех этих случаях мат на клетке какого-нибудь одного цвета. В задаче № 27 на 1. ... gh последует 2. \queen{}f7\mate, а на 1. ... ef 2. \queen{}d5\mate. Нетрудно убедиться, как сходны между собой оба эти мата, которые и называются эхо-матами. Но бывает и так, что картина мата полностью повторяется, но черный король стоит уже на поле другого цвета (см. задачу № 28 -- варианты 1. ... \knight{}~2. \queen{}f5\mate и 1. ... \king{}f6 2. \queen{}e7\mate). Такие эхо-маты называются хамелеонными.

\begin{center}
 \begin{tabular}{ c c }
\textbf{№ 27. А. Челенджер.} & \textbf{№ 28. Г. Готтшаль.}\\
<<The Morning>>, 1897.  & <<Bohemia>>,  1907\\
\chessboard[
\diagramsize,
setfen=2K5/6p1/pQ4p1/5k1N/4p1r1/5N2/8/8,
label=false,
showmover=false]
& 
\chessboard[
\diagramsize,
setfen=n1N5/6Q1/2p1k3/8/7N2K5/8/8,
label=false,
showmover=false] \\
\textbf{Мат в 2 хода (1.\queen{}b7).} & \textbf{Мат в 2 хода (1. \rook{}e2).}
\end{tabular}
\end{center}
	

\textbf{Длинный мат.} Длинными матами называются маты белым ферзем, которые даются им с дальних полей, причем один мат дается по диагонали, а другой по горизонтали. Обычно эти длинные маты ферзем происходят в результате того, что черные заблокировали какие-нибудь поля или перекрыли свои же фигуры. В приведенных двух примерах мы находим эти длинные маты, проведенные с большим искусством.

\begin{center}
 \begin{tabular}{ c c }
№ 29. К. Грабовский & № 30. А. Уайт.\\
I пр. конк. <<Tygodnik іll.>>, 1913.  & I пр. <<G.С.>>, май 1918.\\
\chessboard[
\diagramsize,
setfen=2Q5/8/7B/1pR5/p2k4/1qN5/1rp1Pp2/4nK2,
label=false,
showmover=false]
& 
\chessboard[
\diagramsize,
setfen=8/2pQ2b1/K1Bqp3/8/4R3/N2k4/3P3R/8,
label=false,
showmover=false] \\
\textbf{Мат в 2 хода (1.\rook{}е1).} & \textbf{Мат в 2 хода (1. \rook{}b4).}
\end{tabular}
\end{center}

В задаче № 29 белые своим первым ходом создают угрозу 2. \queen{}с5\mate. на 1. ... \queen{}с4 последует 2. \queen{}h8\mate, а на 1. ... \queen{}:сЗ 2. \queen{}g4\mate. Это и есть <<длинные маты>>. Тонкость первого хода состоит в том, что 1. \rook{}f5 не проходит из-за ответа 1. ... \queen{}:сЗ!

В задаче № 30 <<длинные маты>> проходят в вариантах 1. ... е5 2. \queen{}hЗ\mate! и 1. ... \bishop{}$\sim$ 2. \queen{}h7\mate. На 1. ... \queen{}d5 последует 2. \bishop{}b5\mate, а на 1. ... \queen{}:d7 2. \bishop{}с4\mate. Но самый интересный вариант в этой прекрасной задаче таков: 1. ... \queen{}d4! 2. \rook{}bЗ\mate! (черный ферзь перекрыл слона g7 и заблокировал поле d4).

\textbf{Тема возвращения.} Тема эта состоит в следующем: та же белая фигура, которая двигалась на первом ходу, делает и второй ход, возвращаясь к своему исходному положению. Это движение белой фигуры бывает иногда связано с очень интересными комбинациями.

\begin{center}
 \begin{tabular}{ c c }
\textbf{№ 31. О. 3ахман.} & \textbf{№ 32. М. М. Барулин.} \\
I пр. конк. Герм. Ш. Союза, 1923 & IV пр. конк. <<Известий>>, 1923--25. \\
\chessboard[
\diagramsize,
setfen=n7/3Q4/3PB2K/p5R1/Rq1k4/4p3/N2pp3/2N5,
label=false,
showmover=false]
& 
\chessboard[
\diagramsize,
setfen=3K4/4R1Q1/p1Rp1P1p/r2B3r/1NNk4/6pb/1P1n1qp1/6B1,
label=false,
showmover=false] \\
\textbf{Мат в 2 хода (1.\bishop{}c4).} & \textbf{Мат в 2 хода (1. \rook{}d3).}
\end{tabular}
\end{center}


B задаче № 31 первый ход белых, действительно, очень хорош: помимо того, что черный король получает еще одно свободное поле (с4), белые в то же время развязывают черного ферзя, который может отойти с шахом. Но на 1. ... \queen{}:d6+ - последует 2. \bishop{}е6\mate! Это и есть <<тема возвращения>>, -- белый слон вернулся на свою первоначальную позицию. Грозит 2. \queen{}g4. На 1. ... \queen{}b7 последует 2. \bishop{}b5\mate, а на 1. ... \queen{}b1 2. \bishop{}b3\mate.

Ту же роль, которую в задаче № 31 играет слон, в задаче № 32 выполняет белая ладья. Белые грозят матом \rook{}d3. На 1. ... \queen{}:f6+ последует 2. \rook{}е7\mate! (тема возвращения). Остальные варианты таковы: 1. ... \rook{} d5 2. \knight{}с2\mate, 1. ... \knight{}:с4 2. \rook{}с4\mate, 1. ... \bishop{}f5 2. f7\mate!, 1. ... \rook{}а3 2. \queen{}а7\mate и 1. ... \queen{}:е3 2. \bishop{}:еЗ\mate.

Заканчивая главу о втором (матующем) ходе белых, мы считаем необходимым указать на то, что новая школа стремится наполнить матующий ход определенным идейным и тематическим содержанием. При чтении последующих глав читатель убедится, что во всех темах новой школы (блокирование, развязывание, полусвязывание и пр.) матующий ход имеет первостепенное значение и что именно новой школе удалось придать матующему ходу стратегический характер.
