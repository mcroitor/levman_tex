\chapter{Развязывание.}

 В главе о <<связывании>> мы указывали уже на то, что композиторы чешской школы много поработали в области связывания черных фигур, да еще и поныне продолжают разрабатывать эту тему применительно к трехходовки. Ново-американская школа поставила в центре своего внимания тему <<развязывания>> как черных фигур, так и белых.

Что же такое развязывание? Остановимся сначала на развязывании белых фигур.

В начальном положении задачи (или после первого хода белых) какая-нибудь белая фигура стоит в таком положении относительно своего короля, что лишена возможности двигаться. Например, при белом \king{}а8 и черном \queen{}h1 белый конь стоит на b7. В этом положении белый конь связан. Предположим, что черный конь встанет на d5, сб, f3 или g2. Как только он встанет на линию h1--а8, то белый конь перестает быть связанным или иначе сказать -- развязывается. В двухходовки это развязывание имеет значение постольку, поскольку развязанная фигура, пользуясь своей свободой, дает мат.

Развязывание может произойти, однако, не только в результате того, что какая-нибудь черная фигура встанет на линию связки, а и в результате отхода связывающей черной фигуры (в данном примере -- ферзя h1).

Белая фигура, связанная дальнобойной черной фигурой (ферзем, ладьей, слоном), называется просто связанной. Связывающая черная фигура называется \so{главной}. \so{Главная} фигура может быть свободной в своих движениях, но может быть в свою очередь связана. Возьмем в качестве примера следующую задачу (№ 74):
 
\begin{center} 
 \begin{tabular}{ c c }
\textbf{№ 74. Д. О'Киф.} & \textbf{№ 75. C. С. Лев ман.} \\
I пр. <<G. C.>> дек. 1917. & Журн. <<64>>, 1927 \\
\chessboard[
\diagramsize,
setfen=2K5/4B3/8/1R6/kpQb4/3R4/n1r1r3/3B4,
label=false,
showmover=false]
& 
\chessboard[
\diagramsize,
setfen=2NN2nn/2K2p2/8/Q1B1k3/1pr1P3/2B4b/4RR2/8,
label=false,
showmover=false] \\
\textbf{Мат в 2 хода (1. \rook{}d5).} & \textbf{Мат в 2 хода (1. \king{}b6).}
 \end{tabular}
\end{center}

Белый ферзь связан, он может двигаться лишь по линии связки (то есть по линии с), но в сторону свернуть но может. Главная фигура -- ладья с2 -- в свою очередь связана слоном d1. Развязывание получится только после того, как черный слон или конь встанут на линию с между ладьей и ферзем. И действительно, после 1. ... \bishop{}с5 ферзь получает свободу и дает мат с поля а6. Ha 1. ... \bishop{}с3 последует 2. \queen{}:а2\mate{}, а на 1. ... \knight{}с3 2. \queen{}:b4\mate{}. Такое развязывание, при котором \so{главная} фигура неподвижна, а на линию связки попадают другие черные фигуры, мы называем просто \so{развязываниям}.

В задаче № 75 главной фигурой является ладья с4, свободная в своих движениях, а связанной фигурой является слон. При отступлении ладьи на поля d4 и е4 белый слон развязывается, -- такое развязывание называется прямым. Черная ладья должна двигаться, так как грозит мат 2. \knight{}:с4\mate{}. Ha 1. ... \rook{}d4 последует 2. \bishop{}d6\mate{}, и на 1. ... \rook{}:е4 2. \bishop{}еЗ\mate{}.

Мы уже говорили, что смысл развязывания, соль его, так сказать, заключается в том, что развязанная фигура матует следующим же ходом. В задаче № 74 белый ферзь развязывается трижды и трижды матте, но каждый раз на точно определенном поле. Ha 1. . . . \bishop{}с3 ферзь вынужден матовать с а2, так как слон перекрыл белую ладью d3 и поле bЗ нуждается в защите. После 1. ... \bishop{}с5 ферзь вынужден защищать поле b5, и поэтому мат возможен только с а6. Ha 1. ... \knight{}сЗ ферзь но может матовать ни с а2 ни с а6, но зато конь не защищает больше поля b4, и ферзь получает возможность матовать именно на этом поле.

Все белые фигуры могут быть в роли развязываемых: ферзь, ладья, слон, конь, пешка. Приведем несколько примеров развязывания различных белых фигур и посмотрим, какие комбинации связаны с этой темой.
 
\begin{center} 
 \begin{tabular}{ c c }
\textbf{№ 76. И. Рура.} & \textbf{№ 77. Л. Ротштейн.} \\
II пр. <<G. C.>> 1919 & I пр. <<G. C.>> 1916. \\
\chessboard[
\diagramsize,
setfen=2Nkq2R/p1pB4/8/1Q5b/Kn1p2p1/p7/8/2r5,
label=false,
showmover=false]
& 
\chessboard[
\diagramsize,
setfen=7b/8/4K3/6N1/1R1pQRn1/P1k3r1/3p4/N2brB2,
label=false,
showmover=false] \\
\textbf{Мат в 2 хода (1. \bishop{}e6).} & \textbf{Мат в 2 хода (1. \rook{}f5).}
 \end{tabular}
\end{center}

В № 76 мы находим также развязывание белого ферзя, связанного, однако, не пo горизонтали, как в № 74, a пo диагонали. В начальном положении ферзь свободен, но первым ходом белые сами связывают его, -- это обстоятельство несомненно повышает ценность задачи. Грозит 2. \queen{}d7\mate{}. У черных одна возможность защищаться -- преградить ферзю путь на d7, то есть занять пункт c6. И вот, на 1. ... с6 последует 2. \queen{}а5\mate{}, на 1. ... \knight{}сб 2. \queen{}d5\mate{}, а на 1. ... \rook{}с6 2. \queen{}g5\mate{}. Теперь понятно, почему слон на первом ходу отступил не на f5, а на сб: линия b5--g5 должна быть свободна.

В № 77 мы видим, как белый ферзь, связанный по вертикали, развязывается 4 раза, матуя четырьмя различными способами. Грозит 2. \rook{}с5\mate{}. на -1. ... \bishop{}e5 последует 2. \queen{}сб\mate{}, -- ферзь просто развязан, и другого поля для мата у него нет. Но вот черные сыграли I. ... \knight{}e5. Теперь \queen{}с6 не проходит, нo зато стало возможно 2. \queen{}:d4\mate{}, так как конь перекрыл слона h8. На 1. ... \knight{}еЗ невозможно ни 2. \queen{}с6, ни 2. \queen{}:d4, нo зато получается мат 2. \queen{}dЗ\mate{}, так как конь перекрыл ладью g3. И, наконец, на 1. ... \bishop{}е2 последует 2. \queen{}с2\mate{}, так как слон снял защиту с поля с2. Действительно тонкая и изящная комбинация, делающая честь автору!

\begin{wrapfigure}{r}{0.5\textwidth}
\begin{center} 
 \begin{tabular}{ c }
\textbf{№ 78. Л. Исаев и А. Гуляев.} \\
<<Tijdschrift>>, 1926.\\
\chessboard[
\diagramsize,
setfen=1b3rrq/3p4/6b1/4pN1n/3pk3/2p2R2/B2BnK2/7Q,
label=false,
showmover=false] \\
\textbf{Мат в 2 хода (1. \knight{}e7).} 
 \end{tabular}
\end{center}
\end{wrapfigure}

В следующей задаче (№ 78) мы находим развязывание белой ладьи. В начальном положении ладья свободна, но белые первым ходом сами же связывают её, Защищаясь от угрозы 2. \bishop{}d5\mate{}, черные могут играть 1. ... \bishop{}f7, 1. ... \knight{}f6 и 1. ... \knight{}f4, каждый раз развязывая ладью. Ладья может отойти с матом, но куда? Оказывается, что каждый раз она имеет лишь одно отступление. Ha 1. ... \bishop{}f7 белые должны играть 2. \rook{}gЗ\mate{}, перекрывая черную ладью g8, которая, в противном случае, может помешать мату. Ha 1. ... \knight{}f6 белые отвечают 2. \rook{}h3\mate{}!, перекрывая ферзя h8. При ходе же 1. ... \knight{}f4 черные не только развязывают ладью, но и блокируют поле f4, чем белые и пользуются, матуя ходом 2. \rook{}еЗ\mate{} (двойной шах и мат). Эта комбинация с тройным развязыванием ладьи чрезвычайно трудна и остроумна.
 
Но еще труднее, пожалуй, развязывание белого слона. Только в небольшом количестве задач удалось просто развязать слона в трех вариантах. Одной из таких немногих задач является интересное произведение А. Эллермана (№ 79), в котором есть, однако, существенный недостаток -- двойная угроза матом на g7 или h8. Ha 1. ... с3 белые играют 2. \bishop{}f4\mate{}!, так как слон развязан, но необходимо перекрыть черного ферзя, не допуская его на линию g. Ha 1. ... \knight{}с3 белые играют 2. \bishop{}e1\mate{}, перекрывая ладью а1, а на 1. ... \bishop{}с3 2. \bishop{}е5\mate{}, отрезая черного слона от поля g7. Кроме того, при 1. ... \rook{}:b2 получается прямое развязывание, так как белые отвечают 2. \bishop{}f2\mate{}.
 
\begin{center} 
 \begin{tabular}{ c c }
\textbf{№ 79. А. Эллерман.} & \textbf{№ 80. Г. Броджи.} \\
I пр. <<Brisbane Courier>>, 1919. & I пр. <<G. C.>> 1921. \\
\chessboard[
\diagramsize,
setfen=6k1/2pN3R/3p4/b1p2B2/q1p5/1r4BK/5Q2/rn4R1,
label=false,
showmover=false]
& 
\chessboard[
\diagramsize,
setfen=2Kn4/8/1Q5N/2NpkB2/R7/2r1n3/8/B7,
label=false,
showmover=false] \\
\textbf{Мат в 2 хода (1. \queen{}b2).} & \textbf{Мат в 2 хода (1. \king{}c7).}
 \end{tabular}
\end{center}

Гораздо легче развязывание белого коня, которое проведено в самых разнообразных формах и комбинациях. В № 80 белый конь связан по вертикали. Грозит 2. \queen{}а6\mate{}. Ha 1. ... \knight{}сб белые матуют 2. \knight{}d7\mate{}, так как развязанный конь должен защищать поле f6, которое до хода черных защищал белый ферзь. А на 1. ... \knight{}с4 последует 2. \knight{}с4\mate{}, защищая поле f4. Несложная, но изящная комбинация.

Гораздо ярче и пикантнее комбинация в задаче № 81, идея которой заимствована автором из ранее опубликованной задачи Г. Гвиделли. Грозит 2. \queen{}dЗ. У черных есть две интересных защиты 1. ... c1\knight{}! и 1. ... e1\knight{}! Но в обоих случаях белый конь развязывается и матует: первый раз на f3, а второй -- на е2.
 
\begin{center} 
 \begin{tabular}{ c c }
\textbf{№ 81. К. А. Ларсен} & \textbf{№ 82. Г. Гвиделли.} \\
VIII Амер. Шахм. Конгресс, 1921 & I пр <<G. C.>>, 1917. \\
\chessboard[
\diagramsize,
setfen=4Q3/8/B2N1n2/n2pP3/1P1k4/8/p1pPp2B/r5NK,
label=false,
showmover=false]
& 
\chessboard[
\diagramsize,
setfen=,
label=false,
showmover=false] \\
\textbf{Мат в 2 хода (1. \queen{}g6).} & \textbf{Мат в 2 хода (1. bishop{}b2).}
 \end{tabular}
\end{center}

Остроумнейшую и очень трудную комбинацию с развязыванием коня встречаем мы в задаче 82. Грозит 2. \queen{}d4\mate{}. Предположим, что черные защитились ходом \knight{}f5. Что случилось? Во-первых, белый конь развязан, но кроме того черные заблокировали поле f5. Поэтому белый конь может дать мат с поля f6 (двойной шах), выключая ладью f8. Если черные защищаются ходом c5, то белый конь снова развязан, а поле е5 заблокировано, и конь матует с поля с3 (двойной шах), выключая слона b2. Двойные шахи конь вынужден делать потому, что при ином oтступлении черные спасаются от мата, перекрывая белого слона а8 либо ладьей, либо пешкой. В ответ же на 1. ... с5 (третье развязывание коня) белые играют 2. \knight{}bб\mate{}! Двойные шахи невозможны, но черная пешка не может уже перекрыть слона, а конь отрезает ладью oт поля b7.
    
Одной из лучших задач на тему развязывания белого коня бесспорно является задача № 83. Мы находим здесь 4 развязывания коня. Белые грозят матом 2. \knight{}:f5\mate{}. Ha 1. ... \knight{}d5 последует просто 2. \knight{}е5\mate{}, так как черный конь блокирует поле d5. Ha 1. ... d5 последует 2. \knight{}d6! (необходимо перекрыть черного ферзя). На 1. ... \king{}d5 2. \knight{}с3\mate{}! Но у черных есть еще защита 1. ... \rook{}b7! (на 1. ... \rook{}с7 последует 2. \queen{}:d6\mate{}), которую белые отражают ходом 2. \knight{}f6\mate{}!

\begin{center} 
 \begin{tabular}{ c c }
\textbf{№ 83. П. Тен-Катэ} & \textbf{№ 84. В. Овчинников} \\
I пр. "Grantham Journal", 1926 & II поч. отзыв конк. "64", 1926 \\
\chessboard[
\diagramsize,
setfen=bq1bR3/p2rppP1/1n1pkN2/1P3rP1/1Q2Np2/5K1B/1B2R3/7n,
label=false,
showmover=false]
& 
\chessboard[
\diagramsize,
setfen=b7/b7/4n3/5Q2/4r3/1B6/4PR1/3R2K1,
label=false,
showmover=false] \\
\textbf{Мат в 2 хода (1. \knight{}h5).} & \textbf{Мат в 2 хода (1. \queen{}h3).}
 \end{tabular}
\end{center}

В задаче № 84 проведена очень трудная тема развязывания белой пешки, стоящей на второй линии. Грозит 2. \queen{}d3\mate{}. Ha 1. ... \knight{}d6 последует 2. f4\mate{}, а на 1. ... \rook{}е3 2. f3\mate{}! (пользуясь тем, что поле е3 заблокировано, белые ставят пешку на f3, перекрывая слона а8). Ha 1. ... \bishop{}е3 последует 2. fe\mate{}. Хороши также варианты 1. ... \rook{}d4 2. \queen{}еЗ\mate{} и 1. ... \knight{}f4 2. \queen{}g4\mate{}. Очень изящная задача.
 
\begin{center} 
 \begin{tabular}{ c c }
\textbf{№ 85. С. С. Левман.} & \textbf{№ 86. М. Мэрбл.} \\
Поч. отз. конк. "Шахматы", 1926 & I пр. конк. "G. C.", апрель 1915 \\
\chessboard[
\diagramsize,
setfen=Q1n1bNB1/rr1P3K/pPk3p1/3Rn3/4N2b/4Bpq1/8/8,
label=false,
showmover=false]
& 
\chessboard[
\diagramsize,
setfen=8/5q1p/3pRp2/1p2n3/5B1N/6Nr/QKP2kPn/R7,
label=false,
showmover=false] \\
\textbf{Мат в 2 хода (1. \rook{}a5).} & \textbf{Мат в 2 хода (1. \king{}b3).}
 \end{tabular}
\end{center}
		 
В № 85 мы находим троекратное развязывание белой пешки, связанное с превращением её в разные фигуры и на разных полях. Грозит 2. \bishop{}d5\mate{}. У черных 3 защиты: 1. ... \bishop{}f7, 1. ... \knight{}f7 и 1. ... \knight{}e7, но при этом они развязывают белую пешку. На 1. ... \bishop{}f7 последует 2. dc=\queen{}\mate{}, на 1. ... \knight{}f7 2. de=\queen{}\mate[} (нельзя 2. dc из-за черного ферзя) и на 1. ... \knight{}е7 2. d8=\knight{}!\mate{}, так как черный слон перекрыт.

До сих пор мы рассматривали лишь такие задачи, где развязывается одна какая-нибудь белая фигура. Но мысль композитора на этом не остановилась. Был сделан ряд попыток провести в задаче развязывание двух и даже более фигур. Одной из первых удачных задач этого типа мы считаем задачу № 86. После первого хода сразу связываются две фигуры: ладья е6 и конь g3. Защищаясь от мата пешкой с2, черные могут сыграть 1. ... \knight{}с4, на что последует 2. \rook{}e2\mate{} (ладья развязана), или 1. ... \knight{}е5-f3, на что белые ответят 2. \knight{}hl\mate{}, так как конь теперь развязан.
    
В этой задаче ладья и конь развязываются лишь по одному разу. В задаче № 87 две белые фигуры (ладья и пешка) развязываются каждая дважды. От угрозы 2. \knight{}d4\mate{} черные защищаются отходом коня с5, но, отступая, этот конь развязывает либо ладью, либо пешку. На 1. ... \knight{}d3 последует 2. \rook{}f6\mate{}, на 1. ... \knight{}с4 2. \rook{}с5\mate{}, на 1. ... \knight{}b7 2. с8=\queen{}\mate{}, а на 1. ... \knight{}d7 2. cd=\knoght{}\mate{}! Великолепное произведение!

\begin{center} 
 \begin{tabular}{ c c }
\textbf{№ 87. Г. Венинк.} & \textbf{№ 88. С. С. Левман.} \\
I пр. конк. памяти Шульда, 1922 & "Шахматы", 1927 \\
\chessboard[
\diagramsize,
setfen=5B2/q1P4K/R3k3/1Nn2R2/p2R4/1p2N3/2b2Q2/8,
label=false,
showmover=false]
& 
\chessboard[
\diagramsize,
setfen=3b4/n7/K3K2R3p3p/3pkP2/nR1R3Q/1NpB2p1/5br1,
label=false,
showmover=false] \\
\textbf{Мат в 2 хода (1. \rook{}d4--d5).} & \textbf{Мат в 2 хода (1. \rook{}b4).}
 \end{tabular}
\end{center}

В задаче № 88 развязываются конь и ладья, причем каждая из этих фигур развязывается трижды. Грозит 2. \rook{}xd4\mate{}. Ha 1. ... \knight{}с4 последует 2. \rook{}d3xd4\mate{}, на 1. ... \knight{}а3-b5 или 1. ... \knight{}а7-b5 2. \rook{}е3\mate{}, на 1. ... \knight{}с6 2. \knight{}с5\mate{}, на 1. ... \bishop{}b6 2. \knight{}g5\mate{} и на 1. ... \bishop{}f6 2. \knight{}с5\mate{}.
    
Само собой понятно, что проблемисты ново-американской школы не ограничивались только проведением темы развязывания белых фигур в её чистом виде. Они пытались сочетать ее с другими темами новой школы. Из предыдущих глав этой книги мы знаем уже, что одной из популярнейших тем новой школы была, между прочим, тема шаха на вскрышку белому королю. Можно ли сочетать ее с темой развязывания? Оказывается, что можно, хотя и очень трудно. Приводим в качестве примера 2 задачи:
В задаче № 89, построенной чрезвычайно легко и изящно, эта комбинация проведена в двух вариантах. Белая ладья е2 связана. От мата 2. \queen{}e1\mate{} черные могут защититься шахами на вскрышку: 1. ... \knight{}с4 (d3) + и 1. ... с4+. В обоих случаях, однако, они развязывают ладью, которая и матует, отходя на b2 или е5.
 
\begin{center} 
 \begin{tabular}{ c c }
\textbf{№ 89. .} & \textbf{№ 90. .} \\
. & . \\
\chessboard[
\diagramsize,
setfen=,
label=false,
showmover=false]
& 
\chessboard[
\diagramsize,
setfen=,
label=false,
showmover=false] \\
\textbf{Мат в 2 хода (1. \queen{}b4).} & \textbf{Мат в 2 хода (1. \king{}:a6).}
 \end{tabular}
\end{center}
№ 90. В. Б. Р а й с.
I пр. „G. С.“, апрсль 1917.

Мат в 2 хода (1. CDgl).

В № 90 эта комбинация проведена также в двух вариантах, но при этом развязывается не одна и та же белая фигура, а целых две (1. ... d5+ 2. \knight{}d6\mate{} 1. f5+ 2. \knight{}f6\mate{}).
     
Как мы увидим в дальнейшем, тема простого развязывания сочетается в произведениях современных мастеров с самыми разнообразными темами новой школы, и над этими сочетаниями до сих nop eщe работают виднейшие композиторы нашего времени. Из наиболее выдающихся задач последнего времени, разрабатывающих эту тему, приведем следующие:
 
\begin{center} 
 \begin{tabular}{ c c }
\textbf{№ 91. .} & \textbf{№ 92. .} \\
. & . \\
\chessboard[
\diagramsize,
setfen=,
label=false,
showmover=false]
& 
\chessboard[
\diagramsize,
setfen=,
label=false,
showmover=false] \\
\textbf{Мат в 2 хода (1. \queen{}b4).} & \textbf{Мат в 2 хода (1. \king{}:a6).}
 \end{tabular}
\end{center}
№ 91. Л. Эллорман.
II пр. конк. <<Chcmnitzer Woclien schach>>, 1976.
Мат в 2 хода (1. ФЬЗ).	№ 92. П. Тен-Кпта.
II пр. Шахм. Согоза Голландской Индии, 1926.
Мат в 2 хода (1. Сс14).	 
	
В прекрасной задаче № 91 развязывание белого коня проведено при двух свободных полях у черного короля, что технически очень трудно проводимо. Великолепный первый ход создает угрозу 2. Ag4, на 1. ... с4 последует 2. Кс5, а на 1. ... Лс4 2. Kf2. Интересно то, что развязывание белого коня происходит на одном и том же поле. Очень красив вариант 1. ... cd 2. Ле5.
    
Наилучшее впечатление производит и задача № 92. После первого хода, развязывающего ладью g4, создастся угроза 2. Фе2. Ha 1. ... Ле4 последует 2. Kd2, а на 1. ... Ag2 2. Кс5. Любопытен также вариант 1. ... Лй1+ 2. C:j?l.

В задаче № 93 мы находим некоторые новые моменты: развязываемая белая фигура держит два поля близ короля и матует в результате того, что одно из них блокируется. Таких развязываний с блокированием в задаче три: 1. . . . Се5 2. Ф : Ь7.--1. ... е5 2. Фс4!--1. .. .’CdS 2. Фс2. Любопытны также варианты 1. ... Kg2 2. Kg3 к 1. ... KfS 2. Лс2.
 
\begin{center} 
 \begin{tabular}{ c c }
\textbf{№ . .} & \textbf{№ . .} \\
. & . \\
\chessboard[
\diagramsize,
setfen=,
label=false,
showmover=false]
& 
\chessboard[
\diagramsize,
setfen=,
label=false,
showmover=false] \\
\textbf{Мат в 2 хода (1. \queen{}b4).} & \textbf{Мат в 2 хода (1. \king{}:a6).}
 \end{tabular}
\end{center}
№ 93. С. С. Л е в м а н.
I пр. коіік. „?оік шні Zcit", 1926.
Mar в 2 хода (1. Kh5).	№ 94. Л. A. К с а е в.
II пр. КОНК. „Grantham Journal", 1926,
Ма? в 2 хода (1. Кс5).	 

Задача № 94 интересна тем, что развязывающей фигурой в ней является сам черный король. Угроза здесь 2. Kd7. Ha 1. ... Kpd4 по» следует 2. КЬ51, а на 1. ... Kpf6 2. Л:е6. Достоинство этой задачи заключается также в том, что в начальном положении черный король вообще не имеет свободных полей и получает их после первого хода. Сслмх, Интересна также перемена мата в варианте 1. ... Л: сіб -Ь 2. Kd7 (вместо 2. С : d6).
    
В течение последних нескольких лет знаменитый составитель двухходовых задач, Арнольдо Эллерман, широко пропагандирует новый подход к теме развязывания черной фигуры. Этот новый подход сводится к следующему: во всех приведенных выше задачах развязывание белой фигуры происходит в результате того, что на линию действия главной фигуры становится какая-нибудь другая черная фигура; Эллерман же предлагает обратить внимание на прямое развязывание белой фигуры, т. е. на такое развязывание, которое получается в результате движения главной фигуры, имеющей возможность свободно передвигаться.
    
Пропагандируя новую трактовку темы развязывания, Эллерман указывает на тo, что она делает игру черных (т. е. защиту) более подвижной, гибкой и богатой. Главная фигура, которая при простом развязывании служит лишь для связывания белой фигуры, при прямом развязывании играет выдающуюся роль, защищается самыми разнообразными и остроумными способами. Она не просто развязывает белую фигуру, но при этом играет и сама по себе: блокирует какое-нибудь поле, перекрывает свою или чужую фигуру, развязывает или связывает черную фигуру или, наконец, связывает себя самое,-- и только в результате определенного хода главной фигуры развязанная белая фигура получает возможность дать мат с определенного пункта.
    
Пропаганда Эллермана вызвала большое оживление в среде составителей ново-американской школы, которые усиленно принялись за разработку этой новой области. Особенно много поработали в этой области сам Эллерман, Мари и Гартонг.
    
Для первоначального ознакомления читателя с тем, что представляет собою прямое развязывание, приведем следующую задачу Эллермана, которая по праву считается шедевром автора и одной из лучших двухходовок последнего времени (№ 95).
 
\begin{center} 
 \begin{tabular}{ c c }
\textbf{№ . .} & \textbf{№ . .} \\
. & . \\
\chessboard[
\diagramsize,
setfen=,
label=false,
showmover=false]
& 
\chessboard[
\diagramsize,
setfen=,
label=false,
showmover=false] \\
\textbf{Мат в 2 хода (1. \queen{}b4).} & \textbf{Мат в 2 хода (1. \king{}:a6).}
 \end{tabular}
\end{center}
№ 95. А. Э л л е р м а н.
I пр. „АІгісгс di Re", 1925.
Max в 2 хода (1. Ad7).
	Л1> 96. С. С. Л е в м a it.
I пр. конк. „Шахматы", 1926,
Max в 2 хода (1. Фа4).
	 

Великолепный первый ход, тонкость которого обнаружится в дальнейшем, создает угрозу 2. Фг4. Черные могут защититься от этой угрозы, став ферзем на d4 или с5. Следует, однако, иметь в виду, что черный ферзь является в этой позиции главной фигурой, связывающей белого коня Ь7, и при уходе ферзя с линии Ь конь развязывается и может отойти с двойным шахом. Но отходя с двойным шахом, конь попадает либо на с5, либо на (16, перекрывая при этом одну из белых ладей, и мат не получается. Но вот, когда черный ферзь сыграет 1. ... Ф(14, заблокировав поле е5, белые сыграют 2. Кс5!, выключая ладью а5 и объявляя мат. Ha 1. ... Фсі4 последует 2. Kd6l, выключая ладью (І7.

Перед нами наглядный пример живой и стратегической игры главной фигуры, -- в данном случае игра эта выражается в блокировании.

Мы считаем нужным еще вернуться к этой прекрасной задаче, выяснив значение первого хода. В самом деле, почему белые не сыграли 1. AdS? Оказывается, что у черных есть еще тонкая защита 1. ... ®f2! Теперь двойные шахи конем невозможны, и мат получается лишь при ходе 2. К(І8. Вот почему ладья должна остановиться на с!7!

В задаче № 96 главной фигурой является черный слон, который в начальном положении связан, но после первого хода развязывается и получает возможность играть. Белые грозят ходом 2. Ф<’4, -- защитой от этой угрозы является любое отступление черного слона. Центральные варианты таковы: 1. ... СсЗ 2. Af5 (пользуясь тем, что ладья развязана, а слон перекрыл ладью с2).--1. ... СЬ6 2. Л: d6 (ладья развязана, а слон перекрыл ладью Ь6). В этой задаче прямое развязывание белой фигуры связано не с блокированием (как в предыдущей задаче), а с перекрытием двух черных фигур. Правда, и в этой задаче имеются два варианта с блокированием (1. ... Сс5 2. Ке7--1. . . . Сс5 2. Кс7), но в них мы не имеем момента развязывания белой фигуры.
 
\begin{center} 
 \begin{tabular}{ c c }
\textbf{№ . .} & \textbf{№ . .} \\
. & . \\
\chessboard[
\diagramsize,
setfen=,
label=false,
showmover=false]
& 
\chessboard[
\diagramsize,
setfen=,
label=false,
showmover=false] \\
\textbf{Мат в 2 хода (1. \queen{}b4).} & \textbf{Мат в 2 хода (1. \king{}:a6).}
 \end{tabular}
\end{center}
    № 97. Г. К р и с т о ф ф а н и н и.
VI up, „L'ltalia Sencchistica“, 1926.
Мат в 2 хода (1. {4)
	№ 98. А. Эдлсрма ІІ.
Ill пр. „L/Ilnlin Scncchistica", 1926.

В задаче № 97 мы находим сочетание темы прямого развязывания с темой полусвязки (подробнее о полусвязке см. в следующей главе). Главной фигурой и здесь, как и в предыдущей задаче является слон. Белые грозят матом 2. fe\mate{}, и поэтому слон должен уходить. На 1. ... \bishop{}b8 последует 2. \knight{}d7\mate{}, -- слон перекрыл ладью а8, а черный ферзь после отхода слона связан, поэтому белые получают возможность дать мат на вскрышку, перекрывая своим конем ладью а7. На 1. ... \bishop{}c7 последует 2. \knight{}e8\mate{}! -- повторение той же комбинации. Остальные варианты этой задачи таковы: 1. ... \bishop{}d6 2. \knight{}xd6\mate{}; 1. ... \rook{}xh7\check{} 2. \bishop{}xh7\mate{}; 1. ... \queen{}c1 2. \knight{}d6\mate{}

Ticнтральных вариактах мы находим прямое развязывание белого коня, осложиеиное тем; что г л а в н а я фигура дает шах белому - королю н іі ?о же время свяэквастся. На 1. ... ФП последуот 2, Kf3, a па
1. ... Фсі 2. Ко4. Очень любопытная комбинация.

Новый момент, также весьма интересный, находим мы и в задаче № 99. Белые грозят матом 2. КеЗ (при двух связанных черных фигурах -- ладье ЬЗ и слоне d4). Черным достаточно развязать одну из этих фигур, чтобы угроза отпала. Но на 1. ... Сс4 последует 2. Kf61 а на 1. ... Cd3 2. К : Ь6. Слон Ь5 является здесь главной фигурой. Своими ходами он освобождает коня d7, но при ходе 1. ... Сс4 конь может дать мат лишь с поля f6 (так как развязанная черная ладья держит теперь поле Ьб), а при ходе 1. ... Cd3--лишь с поля Ь6, так как теперь поле f6 охраняется слоном d4.

На этих немногочисленных примерах мы можом убедиться в хом, что тома прямого развязывания действительно таит в себе много интересных и острых комбинаций. Поскольку эта область еще сравнительно мало изучена и разработана, можно полагать, что именно здесь возможны новые интересные достижения.

В заключение приводим несколько задач, иллюстрирующих тему прямого развязывания.
 
\begin{center} 
 \begin{tabular}{ c c }
\textbf{№ . .} & \textbf{№ . .} \\
. & . \\
\chessboard[
\diagramsize,
setfen=,
label=false,
showmover=false]
& 
\chessboard[
\diagramsize,
setfen=,
label=false,
showmover=false] \\
\textbf{Мат в 2 хода (1. \queen{}b4).} & \textbf{Мат в 2 хода (1. \king{}:a6).}
 \end{tabular}
\end{center}
№ 99. A. М а р и и А. Э л л с р м а н.
„Alfierc di Re“, 1926.
Max в 2 хода (1. Кс2).	ЛЬ 100. И. Г а р т о н г.
Ill пр. „Е1 Ajcdrez Argentino", 1926.
Max в 2 хода (1. Кеб).	 

В задаче № 100 тема прямого развязывания проведена в двух вариантах и в каждом из них мы имеем момент блокирования. Грозит 2. Kg7. Ha 1. ... © : сб последует 2. Лс4! (прелестный вариант: белая
слонов--черного и белого), а на 1. ... ладья ггсоекрывает сразу двух ©,:с5 2. Acf4.

В задаче № 101 проводится прямое развязывание белой пешки п двух вариантах. Тема эта содержит технические трудности, счасілшю проодолопныо авгором u приводимой задаче. Первым ходом белые создают угрозу 2. Фg6. Ha 1. .. . Ф : е5 последуеі 2. с14, а на 1. ... Фгі5і 2. сІЗ, так как гсііерь развязан черный слон е4, когорого и нужно псрс- крыть. Остальные вариангы: 1. ... ФсЗ 2. dc.-- 1. ... К: П 2. С : е4.--
1. ... Л: g7 2. Ф : g7.--1. ... fe 2. Ф(2 и 1. ...	2. A:d2.

В задачо № 102, гдо главной фигурой являегся ладья d4, она создаег не 2 идойных варианта, а целых три. Грозит 2. ed. Ha 1. . . . ЛЬ4 последуег 2. Фс7 (слон аЗ перекрьп’ ладьей), на 1. . .. Лс4 2. Фсб (перекрыт черный ферзь) и на 1. ... Ло4 2. ФГ5 (перекрыт слон Ы). Эта труднейшая комбинация проведена в данной задачо очснь легко н изящно.

Заканчивая главу, посвященную теме развязывания белой фигуры, мы считаем необходимым еще остановиться вкратце на таких задачах, где главная фигура хотя и связана сама, но может двигаться по линии связи, развязывая тем самым белую фигуру. Приведем для примера следующие две задачи.

В первой из них (№ 103) главная фигура -- черный ферзь -- может двигаться по линии связки. Грозит 2. d3. При отходе ферзя освобождается поле f5, и угроза, таким образом, не проходит. Ha 1	Ф : Ь7 белые
 
\begin{center} 
 \begin{tabular}{ c c }
\textbf{№ . .} & \textbf{№ . .} \\
. & . \\
\chessboard[
\diagramsize,
setfen=,
label=false,
showmover=false]
& 
\chessboard[
\diagramsize,
setfen=,
label=false,
showmover=false] \\
\textbf{Мат в 2 хода (1. \queen{}b4).} & \textbf{Мат в 2 хода (1. \king{}:a6).}
 \end{tabular}
\end{center}
№ 103. С. С. Л с в м а н.
Похв. отз. „L’ltalia Scacchistica",1926.
Мат в 2 хода (1. Cgl).	№ 104. И. Ритвсльд.
    IV пр. „Chemnitzcr Tagcblatt", 1927.
Мат в 2 хода (1. ЛК4).
	 

 огвотяг 2. f5, но на 1. ... Фг6! этог огвет невозможсн, так как чсриый ферзь окажется развязаниым. Поэгому белые на 1. . . . Ф'/'(> отисчают 2. Kd6, пользуясь тем, что конь развязан, а ладья Ііб перекрыга ферзсм. Осгальные вариаигы эгоіі задачи таковы; 1. ... Кс5 2. К: сЗ (просгое развязывание с перекрыгием ладьи), 1. . . . Ке5 2, fe.
     15 задачо № 104 яга тема проведсна в двух варицагах. Грозит 2. ФоЗ. Иа 1. ... A: g5 или 1. .. . Ле5 белые ответяг 2. Кс4. Иа
1. ... Ad5 эгог мат уже невозможсн, но заго черяые заблокировали поло d5 и сдолали возможным мат 2. Kd7 с выключЬнием бслоіі ладви d7. Остальныв вариангы таковы: 1. ... Kf4 2. ©gl.--1. • . . СсЗ
2. Лс4.--1. ... Коіоо 2. К : d3.
