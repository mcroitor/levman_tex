\chapter{Новейшие двухходовые идеи}

Изучая достижения ново-американской школы в области двухходовой задачи, мы приходим к выводу, что основная работа новой школы в настоящее время сводится к разработке, углублению и расширению тех основных тем, о которых мы подробно говорили в предыдущих главах. Современный автор стремится, например, не только к тому, чтобы провести в задаче несколько развязываний или блокирований, -- его интересует сочетание разнообразных тем и комбинаций, получающиеся в результате этого сочетания.

В процессе разработки и углубления тех основных тем, которые уже знакомы нашему читателю, возникают, однако, порою новые идеи, столь интересные сами по себе, что приковывают к себе особое внимание. В настоящее время, при широкой развитой системе международных конкурсов, эти оригинальные идеи вскоре становятся общим достоянием и подвергаются всесторонней обработке, что быстро обогащает шахматную композицию.

Мы но претендуем в нашей небольшой книге исчерпать все эти идеи, но все же хотим познакомить читателя с наиболее популярными и интересными из них.

Тема Гетгарта. Тема Гетгарта возникла из проработки темы развязывания, осложненной перекрытием черной фигуры. Когда говорят о развязывании, хо обычно имеют в виду развязывание белой фигуры. Гетгарт же посвятил свое внимание развязыванию ч е р н о й ф и г у р ы, причем эхо развязывание имеет место не на первом ходу белых, а на втором, то есть матующем ходу.

Присмотримся внимательнее к задаче № 161, где идея Гетгарта выражена в самой простой форме. При первом взгляде 
на позицию может показаться, что белые сразу могут дать мат ходом Cd5, но это не так, -- когда белый слон попадает на 
d5, то развязывается черный слон, которй и бьёт ладью g3. Отступить же слоном на другое поле нельзя, так как нужно 
отнять у черного короля поле с4. После своего первого хода белые грозят матом 2. Фс2. У черных несколько защит:
1. ... Ло4+, на чхо послодусх 2. Ссб, 1. ... Ло4, на чхо белыо сыграюх 2. Со2. Но ценхральный варканх получихся при защихе
1. ... Kf4, на чхо послодуех 2. Cd5! Эхим ходом беліііо развязываюх чорного слона d6, гю хопсрь это уже но опасно, хак 
как чериый коиь псрекрыл слоиа и ладья g3 не находится уже иод боем. Эхо и есхъ хема Гехгарха.
     В аадаче № 162 эта тема пронодена ужо п диух париантах, что, коночпо, гораядо труднсс и инторсснсе. Сноим 
псрпмм ходом бслыс создают угрозу 2. Кс4. Ha 1. ... Кс5 бслыс отвсгят 2. Ксісб! (раэви- яывая слона Ь6, но нольяупсь том, 
что олон нерекрмт коняі?т), я пп
1. I I I ІШ 2> ІШ!	ЛНДІ/Ю, fc/i'i'HpHtf,	I'btthjft*	j(M I'il
и no моіив? помсшаІІІ магу). Смысл ходоа бізлого (^оіш гганиг оідл иоінпчюй, ссли чорныо сыграют, нанр., і. ... Ка5. Тенсрі» 
нслыш играть конем с!4 ии на сб ІІИ на f3, а ведъ только с этих полей бслый конь может защитить поле g5.
№ 161. С. С. Л е в м а н. „Правда", 1926.
Мат в 2 хода (1. Каі).
	№ 162. Г. Г е -г rap т. Поч. отэ. „Hanclelsblad". 1917.


№ 163. К. X а р л с й. ,,Obscrwcr“, 1926.
Мат в 2 хода (1. Kg5).
	№ 164. С. С. Л е в м а н.
I поч. отз. „Western Morn. News" 1926.
Mar в 2 хода (1. Agl).


В яадачс № 163 та жс 'гсма Готгарта проиодона такжс в двух ва- риантах, но в формо цугцванга, т. с. без угрозы. Своим псрвым ходом бслые меняюг маты в ответ на ходы чсрных 1. ... <W5+ и 1. . . . Qg4. В на- чальном положснии на 1. ... ФГ5 + последовало бы 2. С : f5, а тспсрь белые сыграют 2. К: f5. Нл 1. . . . Ф^4 иослодует 2.’Кк2! аместо С Но ито лштіь побочныо вариаиты. Цонтралыіме жс таковы: 1. ... СГ7
2. КГ51 ((]юрзі> іюрокрыт и но можст понасть иа Іі7) и 1. ... (14 2. Kf3! (тепсрь псрскрыт черный слок). Прекрасная .чадача. ...
      В аадаче № 164 мы находим далыісншсе усложненио тсмы Гст- гарта, которая адссь сочетастся с раавязываішем белой фигурм. В самом дсло, ивоим псршіім ходом бслые соидали угрозу 2. Kf5. При заідпто черных 1. ... Cg4 белыо сыграют 2. KfbS! Этот ход стаиокится поя- МОІКСН не только потому, что ладья 1)4 иерекрыта и ис можсг попас?ь
на гюло f4, но oigo и потому, что в рсзультате хода черного слона конь f5 окавался развязакным. Ha 1. . .. жо Kd4 белыс сыграюг 2. Кеб!, польауясь тсм, что белый конь раавязан, а черный слоіг перокрьгг и иооло того как он будот раавяаан ходом Коб, ои поо !ко ио омоікот іюбіт> слона сЗ.
В качостпо послодного примсрп совромонной обработіш томы Гот- ііарти ііриводим задачу A'u 165. одось гмтарадвское морокрытио сочс-

№ 165. А. П. Г у л я е в.
IV пр. конк. „64“, 1926.
VIат в 2 хода (1КЬ5)	№ 166. А. У a й т.
,Alfіегі сіі Re“, 1925.
Мат в 2 хода (1. ЛГ4).


2. СГЗ. В этой яадаче нужно отмотить то, что черный фсрзь свободен в своих движениях, что дслает тему euje болсо трудной для вопло- щония. |	,
Тикпя жо ію сущоі-ічіу комбишіция, по только с ладьсй, ировсдсііа
в апдачс ваот коин
1. ... К(І6 построения королю.
    Шагом вперед в делс разработки темы Говарда являегся задача № 169. Здесь эта тсма проводсна уже в двух равноцонных вариантах. Белый фераь, двигаясь по линии связки, развязывает чсрного слона,
№ 169. С. С. Л с в м а н.
IV. up. „L’ltalia Scaochistica", 1926
Мат в 2 хода (1. Фіб).
	№ 170. Г. Ю і
. II пр. ,,G. С.‘‘, шоль 1921.
Mar в 2 хода (1. КЬ4).


который может двояким обрааом заіцититься ог угрозы 2. сб. Но на 1. ... Ссб гюследует 2. CC>f2!, а на І. ... Себ 2. Фе5! Таким образом, развязанная черная фигура дважды развязываст болую фигуру. Инте- ресво отметить, что в начальном положснии этой задачи имеются два иллюзорных мага с развязыванием белого ферзя (1. ... Ксб 2. CD: сі5 и 1. ... f6 2. Фс5), которые в действителыюй игре заменяются другой парой развязываиий.
    Тема Юша. Тема Юма содержит в себс комбивацию полусвязки, орга- иичсски сплетенную с развязыванием двух бслых фигур. В качествс при- мера обработки втой трудной темы приводим задачу самого Юма (№ 170). Грозит 2. gf. Черные могут защиЩаться обоими конями, которые стояг в полусвязке. Ha 1. ... Кс4—еЗ последует 2. Ф : f6 (полусвязка плюс развязывание белого ферзя). Ha 1. ... Kd5—еЗ бслые играют 2. Ф : d6 (опять полусвязка с развязываиием ферзя). Тот же мат носледуег и на ход 1. . .. Kd5—1)6. А яа ход 1. ... Kd5—е7 иослсдуег 2. Л: d6 (нолу- связка плюс развязывание ладъи). Таким образом комбинация полусвязкн здссъ дана п сочстании с развязыванием двух белых фигур—ферзя и ладьи. В этом состоит тема Юма.	1 1
    іема сложного перекрытия. Тема сложиого псрекрмгия сгала разра- батьінаться сравнитсльно иедавно. Состоит она в слсдуюіцем: заіциіцаясь от какой-то угрозы, чсрные развязываюх' какую-то св.ою дальнобойную фигуру (ферзя, ладыо), но вместе с тсм и перекрываюг се; белыс используют момент перекрытия и даюі' маг по той линии, по когорой раньшо влияла свяяапная чориая фигура.
    Ііож-иим иа примсрс: в иадачо № 171 era тсма нроводппа днаждьт. Ьолыс грозят матом 2. Cg2. Черпыо могут защититься от агой угроаы
№ 171. Л. ШоР.
I—II np. „Magyar Sakkvilag", 1925..
Mar в 2 хода (1. Фс5).	

лишь косзснкым путем, развязав свою ладыо или ферзя. Возьмем для начала защиту 1. . .. {6. Черныс развязали свою ладью, но одновре- мснно с »гим и псрекрыли ее, так что болые могут дать мат ходом
2. Фсб. Д-Іа 1. ... сб іюслсдует 2. Фсіб, а на 1. ... Кс4 (развяаыванио и псрекрьітис ферзя) 2. Фю5.
Ма
     3 задаче № 172 бслыо грозят ходом 2. с5. Идойныс вариаиты ?гаковы;:.!/ ••• d6 2. Фсб.—1. ... сб 2. <I>d6 и 1. ... Cf6! 2. ЛЬб. Здесь • черньуі ффзь- ^'рижды развязывается и псрокрывается по горизоигали.
; •-3' задаче . 17?.*ясркый ферзь трижды развязъівастся и перекры- ?	no діЦг&#»Ж«. Угроза 2. ЛЬ7. Ыа 1. ... с!4 последует 2. Фв5,
•’П!І 1.	2.	?к ua 1. ... ЛсЗ 2. ft) (тематичоскио варианты).
    Сама по ссбе эга тома но столь трудна: трудно лишь провссти сс одиоврсмснпо а носкольких вариантах. До сих пор удалось добиться лишь троскратиого сложного порокрыткя одиой и той жо чорной фигурм. Приводим двс задачи, где это достигнуто.
	№ 173. Э. Ф у н к.
„The Pittsburg Post", 1925.

г в 2 хода (1. СсІЗ)
Мат 2 хода (1. ЛЬЗ)

вариант
J&I4 послод^е^'2? Ag3, на 1. ... Фе5+- 2. С:е5 и т. д.
N \В задац.О'^ 174 ъ(ы ф?годим дальнсйшую разработку этой интсрсс- \най л;омы. 'З^ідищаясь ?Л.^:,уі;розы 2. Ф<і7, чсрнме развязываюг своого Кс.4, пр?;лодует 2. ФЬ4! Здссь мы имсемг ис только й, пеоФкры?іф» чсркого форзя, но ещо и развязывание гі.-? К?4 белыо огветят 2. Фс4!—чериый фераь фсрзъ/іма^ст, двигаясь по линии связки. И, накопец, іа 1. .. ^Ca^.'^JjbcAQ^yeiP*' 2. Уга4. Трудкость и новизна темы искупают трезращенйс?&4$ІКЯ.ХВ,.г&ерзя ,.jsa порвом ходу.
№ 174. C. C. A e в м a н. Ill rifi. „НШІІЯ Гааста", 1927.
Мат в 2 хода (1. Ф : f7).
	№ 175. Ф. Джсиет.
I np. ,,G. C.", декабрь 1914.
Мат в 2'хода (1. СІ8Ф).


     Тема „пикэниппи1*. Тома „пикэнинпи" состоиг к том, мто чсрная псійка, стоищая па 7-й линии, дсласт чотырс хода, ііричем на каждыіі ход нешки бслые матуют по-разному. ГІриводим две аадачи на эгу тсму.
     В задачс № 175, прииадлежащей Ф. Джснсту, который впсрвмс раарибогал ату '1'ому, бслыо гроаит матом 2. Ф : сб. Чорныо мог?і ааіпи- титься от этой угроаы, уведи пошку с 7-ой линип и сниааи, таким обраном, белого ферзя. Ha 1. ... сЬ последует 2. С : Ь6, на 1. ... сб 2. Кс4, на 1. ... с5 2. С : сЗ и, наконец, на 1. ... cd 2. Cg2.
№ 176. И. Г а р т о н г.
III пр. конк. „Шахматы", 1926.
Мат в 2 хода (1. Фе8).	* , .
	

Очень хорошо проведена также эта занятная тема в задаче № 176. Белые грозят 2. cb\mate{} и этим вынуждают пешку b двигаться. На 1. ... ba последует 2. с7\mate{}, на 1. ... bс 2. \queen{}xс6\mate{} -- это, конечно самые мелкие варианты. Ha 1. ... b5 белые ответят 2. \bishop{}b7\mate{} пользуясь тем, что поле b5 заблокировано. Но самый интересный вариант получается при защите 1. ... b6, на что последует 2. \bishop{}c4\mate{}! (тема Гетгарта). Остались варианты: 1. ... \bishop{}xd3 2. \bishop{}xd3\mate{}, -- 1. ... \queen{}g2 2. \rook{}xd4\mate{}.

На этом мы и заканчиваем главу, посвященную современным темам в двухходовке. Конечно, мы далеко не исчерпали всё то, что могло бы представлять интерес для наших читателей но того, что приведено нами, достаточно для того, любителей в круг идей и замыслов современных проблемистов.
