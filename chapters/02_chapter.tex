\chapter{Ход черных. Защита}

В таких задачах, где первый ход белых создает известную угрозу, черные своим ходом стремятся как-нибудь парировать эту угрозу, защититься от нее. Чем больше возможностей защиты у черных, чем разнообразнее пути защиты, тем труднее белым дать мат на втором ходу и тем содержательнее и интереснее задача.

Защита черных может быть непосредственной, то есть прямой или косвенной. Прямая защита может заключаться в том, что черные организуют прямую защиту того поля или той линии, с которой им грозит мат, или объявляют шах белому королю.

\begin{center}
 \begin{tabular}{ c c }
\textbf{\stepcounter{diagram_counter} № \arabic{diagram_counter}. Г. Альвей.} & \textbf{\stepcounter{diagram_counter} № \arabic{diagram_counter}. А. П. Гуляев.} \\
<<Chess Problems>>, Б. Лоуса. & Поч. отз. конк. <<64>>, 1926. \\
\chessboard[
\diagramsize,
setfen=b1r5/4N2p/n3R3/1p4NK/n2k3p/Q4P2p4b1q/5BB1,
label=false,
showmover=false]
& 
\chessboard[
\diagramsize,
setfen=8/2Kp4/3NbBQ1/pRPk1n1r/r7/pPp1b1N1/2nq4/1B1R4,
label=false,
showmover=false] \\
\textbf{Мат в 2 хода (1.\rook{}е1).} & \textbf{Мат в 2 хода (1. \knight{}е2).}
\end{tabular}
\end{center}

В задаче № 17 мы находим прямую защиту от мата. Белые грозят матом 2.\knight{}е6\mate. Значит, черным нужно как-то защитить поле е6.  Защитить его черные могут различными способами, а именно: 1. ... \bishop{}d5, 1. ... \rook{}с6, 1. ... \knight{}с5 и так далее. Все это прямые защиты, которые легко парируются белыми. 1. ... \bishop{}d5 2.\knight{}f5\mate (поле d5 заблокировано), 1. ... \rook{}с6 2.\rook{}с4\mate (слон а8 перекрыт), 1. ... \knight{}а6--с7 (с5) 2.\queen{}b4\mate (ладья с8 перекрыта), 1. ... \knight{}а4--с5 2.\queen{}b2\mate (ладья с8 перекрыта, а поле с5 заблокировано), 1. ... \queen{}e5 2.\rook{}d1\mate (поле е5 заблокировано), 1. ... \queen{}d6 2.\bishop{}:f2\mate. Но в задаче есть и одна косвенная защита 1. ... \bishop{}еЗ. Она заключается в том, что ладья е1 отрезается от поля с5 и поэтому белые не могут осуществить угрозу мата конем. Они должны сыграть 2.\queen{}:еЗ\mate \footnote{В этой задаче мы находим, однако, и еще одну косвенную защиту 1. ... \rook{}c5! Черные не защищают поля е6, но зато связывают белого коня, который теперь не может отойти с матом. На этот ход белые ответ дают 2.\queen{}d3, так как ладья заблокировала поле с5. Более подробно об этой форме защиты мы поговорим ниже, в связи с задачей № NN}.

Косвенная защита черных гораздо более многообразна и интересна, чем прямая, непосредственная защита угрожаемого поля. В задаче № 18 белые грозят матом 2. К:сЗ. Черные имеют и прямую защиту от неё 1. ... \rook{}с4, но центр задачи в разнообразных косвенных защитах. Во-первых, черным достаточно развязать своего ферзя, чтобы устранить угрозы, -- для этого они занимают какой-нибудь фигурой поле d4. Но на 1. ... \rook{}d4 последует 2. с6\mate, на 1. ... \knight{}с2--d4 2. \bishop{}е4\mate, на 1. ... \knight{}f5--d4 2. \queen{}е4\mate и на 1. ... \bishop{}d4 2. \queen{}g2\mate. Кроме того, черным достаточно отступить слоном е6, так как при этом черный король получает свободное поле. Но на 1. ... \bishop{}f7 белые отвечают 2. \queen{}:f7\mate, а на 1. ... \bishop{}g8 2. \queen{}:g8\mate.

Некоторые иные способы косвенной защиты мы находим в задачах №№19 и 20. В первой из них (№19) белые грозят матом 2. \rook{}с6\mate. У черных есть и прямая защита 1. ... \knight{}d4, на что последует 2. \queen{}:с7\mate, но есть и косвенная защита 1. ... с5!, отнимающая у ферзя поле d4. На этот ход белые ответят матом 2. \queen{}а7\mate. Здесь косвенная защита состоит в перекрытии линии действия белой фигуры. Или же 1. ... \knight{}e5! 2. \queen{}g1\mate.

\begin{center}
 \begin{tabular}{ c c }
\textbf{\stepcounter{diagram_counter} № \arabic{diagram_counter}. Г. Хескот.} & \textbf{\stepcounter{diagram_counter} № \arabic{diagram_counter}. С. П. Крючков.} \\
<<Hampshire Post>>, 1916. & III пр. конк. <<64>>, 1926. \\
\chessboard[
\diagramsize,
setfen=4B2K/4p1Q1/R7/2k5/1p3N2/2b2n2/bN6/2R5,
label=false,
showmover=false]
& 
\chessboard[
\diagramsize,
setfen=7K/3pR3/1p6/1B1k4/3npp2/Bp6/b2QNrp1/b7,
label=false,
showmover=false] \\
\textbf{Мат в 2 хода (1. \bishop{}a4).} & \textbf{Мат в 2 хода (1. \queen{}c3).}
\end{tabular}
\end{center}

В задаче № 20 белые грозят дать мат ходом 2. \queen{}с4\mate. У черных есть интересная косвенная защита -- отступить куда-нибудь конем: тогда белый ферзь окажется связанным. Но при каком-нибудь безразличном отступлении коня может- получиться мат 2. \rook{}е5\mate, -- поэтому черные должны очень точно выбирать, куда отступить конем. Но на 1. ... \knight{}с6 последует 2. \bishop{}с4\mate (поле с6 заблокировано), на 1. ... \knight{}е6 2. \rook{}:d7\mate и на 1. ... \knight{}f3 2. \knight{}:f4\mate. Здесь косвенная защита. состоит в связывании той белой фигуры, которая грозит матом.

В дальнейшем, при рассмотрении идей и тем двухходовой задачи, мы будем иметь возможность познакомиться подробно с различными стратегическими моментами, связанными с защитой черных. Каждая новая защита со стороны черных образует новый вариант, и чем больше стратегических моментов в этой защите, тем выше мы ставим саму задачу. Эти стратегические моменты сводятся к перекрытию или связыванию белых фигур, а также к развязыванию черных фигур. Эти стратегические моменты иногда определяют весь характер задачи.

Есть и такие темы, связанные с ходом черных, которые составляют все содержание задачи, весь стержень ее. Такова, например, \so{тема кольца}, в качестве примера которой мы приводим задачу № 21.

\begin{center}
 \begin{tabular}{ c c }
\textbf{\stepcounter{diagram_counter} № \arabic{diagram_counter}. Г. Хескот.} & \textbf{\stepcounter{diagram_counter} № \arabic{diagram_counter}. Н. Максимов.} \\
I пр. конк. <<Hampstead Express>>, 1905. & <<Шахматный Журнал>>, 1896. \\
\chessboard[
\diagramsize,
setfen=6K1/pN2R1PQ/p7/r2k3r/N2n4/1P2p3/BB5p/2Rb2bq,
label=false,
showmover=false]
& 
\chessboard[
\diagramsize,
setfen=4Q3/1R6/3p4/3k1NK1/1P6/3p4/3P4/8,
label=false,
showmover=false] \\
\textbf{Мат в 2 хода (1. \bishop{}a4).} & \textbf{Мат в 2 хода (1. \queen{}c3).}
\end{tabular}
\end{center}	

Тема кольца заключается. в том, что черный конь имеет возможность, в целях защиты, описать на доске полный круг, т. е. встать на любое из доступных ему полей, причем на каждый его ход следует особый мат. В задаче № 21 тема эта проведена блестяще. Белые грозят 2. \knight{}сЗ\mate, -- при отходе коня d4 эта угроза не проходит. На 1. ... \rook{}с2 последует 2.b4\mate, на 1. ... \knight{}е2 2.\queen{}:h5\mate, 1. … \knight{}f3 2. \queen{}e4\mate, 1. ... \knight{}f5 2. \rook{}e5\mate, 1. ... \knight{}е6 2. \rook{}e7-d7\mate, 1. ... \knight{}c6 2. \rook{}c7-d7\mate, 1. ... \knight{}b5 2. \rook{}c5\mate, 1. ... \knight{}:b3 2. \queen{}d3\mate. Конь d4 описал полный круг.

В этой задаче мы видим максимальную свободу движений черного коня. Эта же тема может быть применена и к прочим черным фигурам: дать им, в целях защиты, максимум свободы передвижения -- для слона, ладьи, ферзя, пешки и даже короля. Так, в задаче №22 мы видим, как черному королю дана возможность отступить на одно из 5 свободных полей (есть задачи, где король имеет 6 и даже 7 свободных полей). Мы не будем подробно останавливаться на этих темах и укажем лишь, что здесь соприкасаемся с областью так называемых \so{рекордных} задач (максимумов). В этих задачах автор стремиться выразить какую-нибудь тему в максимальном количестве вариантов, - примером этой темы является и задача №22. Иногда даются и такие задания: подставить белого короля под возможно большее число шахов, дать белым ферзем возможно большее число матов и т.д. Однако увлечение рекордными задачами значительно ослабело сейчас, когда к задаче подходят не только с формальными требованиями.

Ново-американская школа, в противовес к прежним, обратила сугубое внимание на момент защиты. Она выдвинула положение, что задача должна состоять не только из остроумной и тонкой игры белых, но также из тонкой и изобретательной игры черных. Это внесло в двухходовку целый ряд стратегических идей, которые чрезвычайно оживили эту область задачной композиции. Новая школа сделала черных, то есть защищающуюся сторону, равноправной борющейся стороной -- и в этом ее главнейшая заслуга.
