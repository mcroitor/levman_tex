\chapter*{ВСТУПЛЕНИЕ}
\addcontentsline{toc}{chapter}{ВСТУПЛЕНИЕ}
Двухходовкой называется задача, в которой требуется дать мат в 2 хода. Начинают белые. Таким образом, задача эта состоит в сущности из 3-х ходов: первого хода белых, ответа черных (защита) и второго хода белых (мат). Правда, защит в задаче может быть несколько, в зависимости от чего колеблется и число матов в задаче.

Количество белых и черных фигур в задаче ограничено так же, как и в шахматной партии: у каждой стороны может быть 16 фигур. Все правила шахматной игры сохраняются и в задаче: рокировка, взятие en passant, право превращения пешки в любую фигуру и так далее. В задаче также не должно быть таких положений, которых теоретически нельзя получить из партии. Например, невозможно расположение белых пешек на а2, а3 и b2, белого слона на а1 при наличии белой пешки b2 и так далее. Это относится ко всем задачам и, в частности, и к двухходовке.

Конечно, мы редко встречаем в двухходовке такую позицию, в которой участвуют все 32 фигуры. Обычно в задаче имеются лишь те фигуры, которые необходимы для проведения идеи или темы задачи. Тема или идея -- это то содержание, которое автор пытается вложить в задачу. Изучению тем и идей двухходовой задачи и посвящена первая часть нашей книги.

Белых фигур в задаче обычно бывает лишь столько, сколько безусловно необходимо для проведения данной темы, а также для того, чтобы задача решалась, то есть иными словами для того, чтобы мат получался в ответ на любую защиту черных. Черных же фигур бывает столько, сколько их необходимо для проведения защиты. Лишних фигур, -- т. е, таких, которые не принимают никакого участия в игре, которые не нужны ни для атаки, ни для защиты, ни для решения, -- в задаче не должно быть.

Такая задача, где имеется всего не более 7 фигур (напр. 4 белых и 3 черных) называется миниатюрой (смотрите, например, задачу № 141). Задача же с числом фигур не более 12 называется мередит (см. задачи №№ 68, 71, 74, 84).

Есть двухходовые задачи на \so{угрозу} и на \so{цугцванг}. К первому виду принадлежат такие задачи, где белые, после своего первого хода, грозят каким-нибудь матом в том случае, если черные сделают какой-нибудь безразличный ход, то есть не будут защищаться. Цугцванг (Zugzwang) же есть такое положение, где белые никаким определенным матом после первого хода не грозят, так как у черных все защищено, -- но так как черные должны ходить, то они сами ухудшают свое положение, и белые получают возможность матовать на 2-м ходу.

Для того, чтобы читатель мог детальнее ознакомиться с содержанием и построением двухходовой задачи, мы сначала рассмотрим, подробнее, что такое первый ход белых (ключ к задаче), ход черных (защита) и 2-ой ход белых (мат), а затем перейдем к изучению тем и комбинаций современной задачи.