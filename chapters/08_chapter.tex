\chapter{Полусвязывание}

Тема полусвязывания появилась и стала разрабатываться сравнительно недавно, немногим больше 10 лет тому назад. Собственно говоря, комбинация, лежащая в основе этой темы, была и ранее известна, но на нее не обращали никакого внимания, -- до той поры, пока ново-американская школа не показала, как много нового и интересного таит в себе эта тема.

Состоит эта тема в следующем. На одной линии с черным королем стоит дальнобойная белая фигура (ферзь, ладья, слон), а между ними стоят какие-нибудь две черные фигуры. Пока обе фигуры стоят на этой линии, каждая из них может свободно двигаться и уйти с этой линии. Но вот одна из этих фигур, в целях защиты или просто в силу необходимости сделать ход, сошла с этой линии. Что сталось со второй черной фигурой? Она оказалась связанной. То же случилось бы с первой черной фигурой, если бы отошла вторая. Таким образом, каждая из них как бы полусвязана. Комбинация же состоит в том, что при отходе одной из черных фигур белые имеют возможность матовать только потому, что оставшаяся на линии черная фигура связана. Поясним примером:

\begin{center} 
 \begin{tabular}{ c c }
\textbf{№ 105. К. Грабовский} & \textbf{№ 106. К. Мэнсфильд} \\
II пр. <<G. С.>>, март 1916 & . <<British Chess Magazine>>, 1922\\
\chessboard[
\diagramsize,
setfen=6Nk/3R1Bnr/7p/8/b7/2r5/2n3Qb/B6K,
label=false,
showmover=false]
& 
\chessboard[
\diagramsize,
setfen=6K1/6p1/7b/4Pk2/2Q2n1P/2B3R1/5nP1/5R2,
label=false,
showmover=false] \\
\textbf{Мат в 2 хода (1. \knight{}e7).} & \textbf{Мат в 2 хода (1. \rook{}g6).}
 \end{tabular}
\end{center}
 
В задаче № 105 линией полусвязки является диагональ а1--h8. Между белым слоном и черным королем стоят две черные фигуры, ладья и конь: Если двинется ладья, то конь окажется связанным; если же отойдет конь, то связанной окажется ладья. Таким образом, мы имеем в этой задаче такую позицию, которая может привести к полусвязыванию ладьи или коня. Но пройдет ла эта комбинация, мы еще не знаем. Это мы увидим лишь в процессе решения.

Итак, после первого хода белые грозят 2. \knight{}g6\mate{}. Как защищаться черным? Двигать коня д7 нехорошо, так как на любой его отход последует 2. \queen{}g8\mate{}. Значит, нужно защищаться ладьей. На 1. ... \rook{}g3 последует 2. \queen{}а8\mate{}! Почему стал возможен этот мат? Во-первых, потому, что при отходе ладьи конь связан (это и есть полусвязка!), но кроме того еще и потому, что ладья перекрыла слона h2. Если же ладья защититься отмата ходом \rook{}с6, то белые сыграют 2. \rook{}d8\mate{}! -- снова полусвязка и перекрытие другого черного слона (при 1. ... \rook{}eЗ 2. \rook{}d8 не проходит из-за ответа 2. ... \bishop{}e8). Отсюда мы видим, что комбинация полусвязывания проходит в этой задаче дважды.

Но комбинация эта проходит лишь при движении одной из черных фигур, стоящих в полусвязке, ладьи, -- при движении ;ко коня связка ладьи не имеет значения, т. е. комбинация полусвязывания не проходит. Такое полусвязывание, которое проходит лишь в результате движения одной из полусвязанных фигур, называется \so{неполным}.

В задаче № 106 мы видим уже \so{полное} полусвязывание. Задача эта, построенная на цугцванге, принадлежит родоначальнику темы полусвязывания, английскому композитору К. Мэнсфильду. Мэнсфильд дал ряд великолепных образцов обработки этой темы, некоторые из которых мы приведем в дальнейшем. В указанной выше задаче в полусвязке стоят два черных коня. При любом движении коня f4 (кроме 1. ... \knight{}xg6) белые матуют ходом 2. \queen{}g4\mate (при связанном коне f2), а на 1. ... \knight{}xg6 последует 2. g4\mate (опять полусвязка). При любом отходе коня f2 (кроме 1. ... \knight{}e4) белые матуют ходом 2. \queen{}dЗ\mate (при связанном коне f4), а на 1. ... \knight{}е4 последует 2. \queen{}е6\mate (четвертая полусвязка). В этой задаче комбинация получается в результате движения обеих полусвязанных фигур, т. е. является полной.

Уже из этих первых задач на тему полусвязывание мы можем сделать один существенный вывод: комбинация полусвязывания редко проходит в своем элементарном виде, т. е. редко бывает, чтобы одна из полусвязанных черных фигур, покидая линию полусвязки, \so{только} связывала вторую черную фигуру. Обычно, она при этом делает еще кое - что: перекрывает какую-нибудь другую фигуру, блокирует поле, дает мат и пр. Таким образом, тема полусвязывания сочетается с другими темами ново-американской школы. В дальнейшем, рассматривая различные комбинации с полусвязыванием, мы будем указывать, с какими идеями переплетается тома полусвязывания.

Какие черные фигуры могут создать комбинацию полусвязки? Все, начиная от ферзя и кончая пешкой. При этом этом могут быть и одинаковые фигуры (два коня, две ладьи) и различные (конь и ладья, слон и пешка и пр.), а полусвязывание может происходить и по диагонали и по линии (горизонтальной или вертикальной).

Рассмотрим несколько случаев полусвязывания одинаковых черных фигур.

\begin{center} 
 \begin{tabular}{ c c }
\textbf{№ 107. С. Млотковский} & \textbf{№ 108. А. Мари, А. Эллерман} \\
III поч. отзыв <<G. C.>>, 1919 & II приз, <<Brisbane Courier>>, 1926 \\
\chessboard[
\diagramsize,
setfen=2K5/4B3/8/1R6/kpQb4/3R4/n1r1r3/3B4,
label=false,
showmover=false]
& 
\chessboard[
\diagramsize,
setfen=B2N1B2/4r3/1p1r1pp1/2k2q1R/Pp5p/1P2P3/3R1N2/n4KQ1,
label=false,
showmover=false] \\
\textbf{Мат в 2 хода (1. \king{}a2).} & \textbf{Мат в 2 хода (1. \queen{}h2).}
 \end{tabular}
\end{center}

В первой из этих задач (№ 107) мы имеем диагональное полусвязывание двух ладей в самом элементарном виде. В самом деле, защищаясь от мата ферзем на bЗ, черные могут играть либо 1. ... \rook{}e2 на что последует 2. \queen{}xе2\mate , либо 1. ... \rook{}d3 2. \queen{}с6\mate{}, либо 1. ... \rook{}с5 2. \queen{}e4\mate{}. Конечно, при таком ограниченном количестве белых и черных фигур провести эту тему полнее весьма трудно. Но вот в задаче № 108 мы находим прекрасную обработку этой темы, соединенной с другими темами новой школы.

Белые грозят 2. \queen{}d6\mate{}. Черные могут защищаться обеими ладьями. Но на 1. ... \rook{}e7 -- d7 (или е6) последует 2. \knight{}e6\mate{}. Интереснее игра второй ладьи: 1. ... \rook{}d6xd2 (или d3 и пр.) 2. \queen{}с7\mate, 1. ... \rook{}d6d7 2. \knight{}e6\mate. 1. ... \rook{}с6! 2. \knight{}b7\mate. В последнем варианте мы имеем сочетание полусвязывания с блокированием. Эта игра двух черных ладей, стоящих в полусвязке, соединена в задаче с игрой черного ферзя: 1. ... \queen{}e5 2. \knight{}е4\mate! и 1. ... \queen{}d5 2. \knight{}d3\mate! (развязывание белого коня плюс перекрытия черных ладей). Обе темы так тесно переплетены в задаче, что трудно сказать, какая из них является основной -- полусвязывание или развязывание.

Тему горизонтального полусвязывания двух черных ладей мы находим в прекрасной задаче Эллермана (№ 109). Белые грозят 2. \queen{}f4\mate. Идейные защиты черных таковы: 1. ... \rook{}с4 2. \knight{}d3\mate (полусвязывание плюс перекрытие слона а6), 1. ... \rook{}сЗ 2. d4\mate (полусвязывание плюс перекрытие слона а1) и 1. ... \rook{}d4 2. \knight{}с6\mate (полусвязывание). Задача имеет еще 'несколько вариантов, но уже без полусвязывания.

Прелестная комбинация с полусвязыванием двух слонов (конечно, в отношении слонов возможно лишь горизонтальное или вертикальное полусвязывание) проведена в задаче № 110. Правда, комбинация полусвязки проходит только в двух вариантах, но варианты сами по себе великолепны. Грозит 2. \rook{}с7\mate. Ha 1. ... \bishop{}d5! последует 2. \queen{}h7\mate! (полусвязка плюс перекрытие ладьи) и на 1. ... \bishop{}d6 2. \queen{}g4\mate (полусвязка плюс перекрытие той же ладьи). Только найдя оба эти варианта, можно понять смысл и тонкость первого хода. Слон должен отступить именно на h2. так как на e5 или f4 он будет мешать двигаться белому ферзю в тематических вариантах.

\begin{center} 
 \begin{tabular}{ c c }
\textbf{№ 109. А. Эллерман} & \textbf{№ 110. Г. Гвиделли} \\
II пр. <<G. C.>>, ноябрь 1917 & I пр. <<G. C.>>, январь 1917\\
\chessboard[
\diagramsize,
setfen=3N1n1q/2n5/b2p2pp/R1rrk3/1N4P1/5Q2/3P2pK/b5B1,
label=false,
showmover=false]
& 
\chessboard[
\diagramsize,
setfen=K1k5/nRB2N2/P7/2b5/prbQ3p/3r1p2/8/2Rq4,
label=false,
showmover=false] \\
\textbf{Мат в 2 хода (1. \king{}g3).} & \textbf{Мат в 2 хода (1. \bishop{}h2).}
 \end{tabular}
\end{center}

В задаче Г. Гвиделли мы имеем лишь два варианта с полусвязывания. В № 111, принадлежащем знаменитому итальянскому композитору Альберто Мари, эта трудная тема проведена в четырех прекрасных вариантах.

\begin{center} 
 \begin{tabular}{ c c }
\textbf{№ 111. А. Мари} & \textbf{№ 112. Ф. Дженет} \\
I пр. конк. <<Brisbane Courier>>, 1922 & пр. <<G. C.>>, май 1919 \\
\chessboard[
\diagramsize,
setfen=3n4/1p1K4/B2p4/3k1rQ1/1Pp4R/3b4/2rbN3/B2R4,
label=false,
showmover=false]
& 
\chessboard[
\diagramsize,
setfen=b3K3/1BP1R2p/2n2P2/3nBp2/1rP1kP1r/1pPR4/1N4p1/2Q3b1,
label=false,
showmover=false] \\
\textbf{Мат в 2 хода (1. \bishop{}f6).} & \textbf{Мат в 2 хода (1. \rook{}g3).}
 \end{tabular}
\end{center}

Грозит 2. \rook{}d4\mate. Черные могут защищаться обоими слонами. На 1. ... \bishop{}e4 последует 2. \knight{}f4\mate, так как слон d2 связан, а поле e4 заблокировано, что дает возможность выключить ладью h4. Ha 1. ... \bishop{}сЗ последует 2. \bishop{}xс4\mate -- опять полусвязывание, на этот раз с перекрытием ладьи с2. На 1. ... \bishop{}f4 белые ответят 2. \queen{}g2\mate!, пользуясь полусвязкой слона, а также перекрытием ладьи f5, а на 1. ... \queen{}еЗ 2. \queen{}f5\mate. Тонкость первого хода становится ясна из варианта 1. ... \knight{}с6 2. \queen{}g8\mate, -- необходимо первым же ходом перекрыть линию f, чтобы отрезать черную ладью от поля f7.

Примером вертикального полусвязывания двух коней может служить приведенная выше задача № 106. Диагональную полусвязку их мы находим в задаче № 112. Угроза здесь 2. ФЫ. Главную игру дает конь d5. Ha 1. ... Кс7~Ь последует 2. С : с7, на 1. . . . К : f6 4- 2. С : S6, ка
1. ... К: f4 2. С: f4, а на 1. ... КеЗ 2. Cd6. Ha 1. ... Kd4 также последует 2. С : <14, а на 1. ... К: сЗ 2. С: сЗ.

Очень трудна тема полусвязывания двух пешек. Все же есть немало задач, где тема эта хорошо проведена. В качестве примера приведем задачу известного датского композитора К. A. К. Ларсена (№ 113): в ней полусвязывание черных пешек переплетается с моментами их превращения в различные фигуры. Защищаясь от угрозы 2. КеЗ, черные могут сыграть 1. ... flKl, на что последует 2. Лсі. При защите 1. . . . dlK белые играют 2. Kel, а при ходе 1. следуег 2. Фс7.

\begin{center} 
 \begin{tabular}{ c c }
\textbf{№ 113. .} & \textbf{№ 114. .} \\
. & . \\
\chessboard[
\diagramsize,
setfen=,
label=false,
showmover=false]
& 
\chessboard[
\diagramsize,
setfen=,
label=false,
showmover=false] \\
\textbf{Мат в 2 хода (1. \rook{}d5).} & \textbf{Мат в 2 хода (1. \king{}b6).}
 \end{tabular}
\end{center}
№ 113. К. A. К. Ларсои.
IV пр. „G. С.“, ноябрь 1920.
Мат в 2 хода (1. Kg4).	№ 114. К. В э т н е й.
II up. ,,G. С.“, май 1920.
Mar в 2 хода (1. Ag8).

Познакомившись с полусвязыванием одинаковых черных фигур, мы переходим к тем комбинациям, которые возникают в результате полу- связывания различных фигур. Исчерпать все эти сочетания мы, конечно,

\begin{center} 
 \begin{tabular}{ c c }
\textbf{№ 115. .} & \textbf{№ 116. .} \\
. & . \\
\chessboard[
\diagramsize,
setfen=,
label=false,
showmover=false]
& 
\chessboard[
\diagramsize,
setfen=,
label=false,
showmover=false] \\
\textbf{Мат в 2 хода (1. \rook{}d5).} & \textbf{Мат в 2 хода (1. \king{}b6).}
 \end{tabular}
\end{center}
№ 118. K. A. K. Л a p c e II. II up. ,,G. C.“, фсврпль 1920.
Мат в 2 хода (1. Kf3).
	Мат в 2 хода (1. Kpc5)

B задаче № 118 мы видим сочетание уже трех тем: полусвязки, развязывания и еще шахов на вскрышку. Такое сочетание чрезвычайно трудно осуществимо, но в этой задаче оно очень хорошо и искусно представлено. Грозит 2. ФГ4, но первым своим ходом белые связали коня с15 и сделали возможными открытые шахи конем со стороны черных. Но на безразличный шах конем (напр. 1. ... Ка5 + ) последует просто 2. С: с2ч (полусвязка). Поэтому черные должны так отойти конем, чтобы одновременно защититься от этого мата. Для этого им нужно попасть конем на е5. Но на 1. ... Кс5 последует 2. КсЗ! (развязывание с полусвязкой). Ha 1. . . . d3-f-6eAbie отвечают 2. Л: еЗ (полусвязка), На 1. ... е5 последует 2. Kf6. Прекрасная и трудная задача!

\begin{center} 
 \begin{tabular}{ c c }
\textbf{№ 117. .} & \textbf{№ 118. .} \\
. & . \\
\chessboard[
\diagramsize,
setfen=,
label=false,
showmover=false]
& 
\chessboard[
\diagramsize,
setfen=,
label=false,
showmover=false] \\
\textbf{Мат в 2 хода (1. \rook{}d5).} & \textbf{Мат в 2 хода (1. \king{}b6).}
 \end{tabular}
\end{center}
№ 120. Л. А. И о а о іі. Конкурс жури. „64", 1925. (Исправлснпая).

В задаче № 119 к этим трем темам добавлен еще один момент: развязывание черной фигуры. В самом деле, после 1-го хода белых грозит 2. f4. Как защищаться черным от этой угрозы? Прежде всего, путем шаха. Но на 1. ... с6 + или 1. ... cb-f- последует просто 2. Л: е7 (полусвязка). Поэтому черные играют 1. ... с5+, развязывая черного коня d5 и препятствуя указанному выше мату, Но своим ходом черные развязали также белого коня Ьб, который и дает мат 2. Kd7! Итак, в варианте 1. ... с5~Ь 2. Kd7 мы имеем четыре момента: шах на вскрышку, развязывание белой фигуры, развязывание черной фигуры и полусвязку. Изумительное достижение! Правда, задача получилась весьма громоздкой, но ведь и замысел исключительной трудности. Есть в задаче и еще один красивый вариант 1. ... Ko5 (развязывая черного коня c!5) 2. Kc4 (второе развязывание белого коня).

Великолепную обработку сходной, но еще более оригинальной темы мы находим в задачс № 120. Белые грозят матом Kd6. Защититься от этой угрозы белые могут разными способами, -- в первую очередь, развязывая своего ферзя. Но на 1. ... Cd5 последует 2. Фс!3!, так как белый ферзь развязан, а конь Ь4 стоит в полусвязке. Таким образом, мы имеем здесь сочетание трех тем: развязывания черной фигуры, развязывания белой фигуры и комбинации полусвязки. Но это же сочетание проведено еще в одном варианте: 1. ... Kd5 2. ФЫ! Прочие варианты: 1. ... Kpf5 2.  ; с5. -- 1. А.: сб 2. K:g5.

\begin{center} 
 \begin{tabular}{ c c }
\textbf{№ 119. .} & \textbf{№ 120. .} \\
. & . \\
\chessboard[
\diagramsize,
setfen=,
label=false,
showmover=false]
& 
\chessboard[
\diagramsize,
setfen=,
label=false,
showmover=false] \\
\textbf{Мат в 2 хода (1. \rook{}d5).} & \textbf{Мат в 2 хода (1. \king{}b6).}
 \end{tabular}
\end{center}
№ 121. К. Ш с п п а р д. 
III пр. „G. С.“, май 1919.
Мат в 2 хода (1. ФЬ2).
	№ 122. А. П. Г у л я е в.
„Шахматы", 1927.
Мат в 2 хода (1. Kpd7).

До сих пор мы знакомили наших читателей с такими задачами, где проводится лишь одна система полусвязывания (системой полусвязывания мы называем позицию, где между дальнобойной белой фигурой и черным королем стоят две черные фигуры, которые своей игрой создают комбинацию полусвязки). Но таких систем в одной задаче может быть и больше. Для примера берем задачу № 121. Здесь две таких системы: первая по линии g, а вторая по третьей линии. Задача построена на Zugzwang. Тематические варианты таковы: 1. ... Ag4 СЛ 2. К : !і5 -- 1. ... Лс5 2. © : с5. -- 1. ... Ла5 (Ь5, с5 и т. д.) 2. Ке4, -- 1. ... f2-i- 2. © : f2.—1 ... с2 2. К : е2. Итого -- пять полусвязок. На 1. .. . Кр.Ч последует 2. ФЬ8.

Полусвязка может быть не только открытой, но и скрытой, замаскированной. Пример такой замаскированной системы полусвязки мы находим в задаче Лга 122. Белые грозят 2. Кс7. Черные в целях защиты должны играть конем Ь5, связывая коня сб. Но на 1. ... Kd4 последует 2. © : (7 (развязывание с полусвязкой), а на 1. ... Kdo 2. ФГЗ (то же самое). Полусвязка обнаруживается здесь лишь в самый последний момент, -- после первого хода белых она находится в замаскированном виде.

Заканчивая главу о полусвязывании, мы считаем необходимым указать на то, что мы и в малой степени, не исчерпали всех комбинаций, скрытых в этой интереснейшей теме. Хотя на поприще полусвязывания сделано и достигнуто ужо очень много, но мы отнюдь не считаем эту область новой школы исчерпанной. Здесь возможна еще большая. и продуктивная работа.

В заключение помещаем две задачи К. Мэксфильда, где приведено максимальное количество полусвязываний при интересной и острой игре.

\begin{center} 
 \begin{tabular}{ c c }
\textbf{№ 123. К. Мэнсфильд} & \textbf{№ 124. К. Мэнсфильд} \\
1 пр. конк. <<Hampshire Post>>, 1926 & 1 пр. конк. <<El Ajedrez Argentino>>, 1915 \\
\chessboard[
\diagramsize,
setfen=2K5/4B3/8/1R6/kpQb4/3R4/n1r1r3/3B4,
label=false,
showmover=false]
& 
\chessboard[
\diagramsize,
setfen=2NN2nn/2K2p2/8/Q1B1k3/1pr1P3/2B4b/4RR2/8,
label=false,
showmover=false] \\
\textbf{Мат в 2 хода (1. \rook{}d5).} & \textbf{Мат в 2 хода (1. \king{}b6).}
 \end{tabular}
\end{center}

Замечательная задача (№ 123) была одним из первых примеров обработки полного полусвязывания и вызвала к жизни большое движение в области шахматной композиции. В этой задаче мы находим 6 полусвязок, а именно 1. ... (защищаясь от угрозы 2.) 2. ; -- 1. ... 2. (выключая слона ); 1. ... или 2.

