\chapter*{ОТ АВТОРА}
\addcontentsline{toc}{chapter}{ОТ АВТОРА}
Шахматная задача есть искусственная позиция со специальным заданием: белые, начиная, дают мат в указанное количество ходов. Черные могут защищаться всеми доступными им средствами, но, тем не менее, белые дают мат в определенный срок.

К оценке задачи нельзя подходить с теми требованиями и взглядами, которые обязательны при рассмотрении всякой обычной позиции, взятой из практической партии. В партии, а также в этюде, мы учитываем прежде всего соотношение сил и ищем выигрыша для той или другой стороны. Если у белых, скажем, перевес в силах и позиции, то мы признаем, что белые должны выиграть: в партии путей для выигрыша бывает обычно много, в этюде имеется лишь один путь к выигрышу. В задаче же нас соотношение сил не интересует: пусть у черных будет один король, а у белых все фигуры — нас это смутить не может. Мы обращаем внимание главным образом на позицию черного короля и ищем путей к мату в определенное количество ходов. Как мы дальше увидим, в каждой задаче есть лишь один путь решения, который приводит к цели при любой защите со стороны черных.

Нужно ли доказывать право шахматной задачи на существование? Нам кажется это излишним. Большинство шахматистов задачами не интересуется, предпочитая практическую игру, но это предпочтение — дело вкуса. Партия и задача—две совершенно самостоятельные области шахматного искусства, не поддающиеся даже сравнению. Каждая из них по-своему привлекательна и каждая разрабатывает особые проблемы.

Шахматная задача, как и вся шахматная игра, имеет свою историю: от первых <<загадок на пари>>, взятых из положения, случившегося в партии, до высоко художественных, сложных и трудных произведений, содержащих ряд тонких и глубоких стратегических идей и эффектных комбинаций.

Мы не намерены, однако, дать в этой книге очерк развития шахматной задачи. Наша задача скромнее: познакомить любителей с основными темами и идеями современной задачи, с техникой составления задач и достижениями, имеющимися в этой области шахматного искусства. Но читатель должен иметь в виду, что эти темы и достижения возникли не сразу, а в результате долгой творческой работы сотен проблемистов во всех странах мира. Десятками лет накоплялся опыт, знания, умения, создавались школы и направления, строилась система шахматной композиции. Рассматривая темы и идеи современной задачи, мы будем попутно касаться и истории развития задачи, знакомя читателей, с отдельными школами к крупнейшими мастерами задачного искусства.

В первой части книги мы будем говорить о двухходовой задаче, во второй части — о трехходовой и многоходовой
Если этот труд хоть в небольшой степени поможет начинающим любителям ориентироваться в вопросах шахматной композиции, если он будет содействовать продвижению задачного искусства в широкие круги советских шахматистов, — автор будет считать свою задачу выполненной.
\newline\newline
\textbf{Москва, 25 июня 1927.}
